

% --------------------------------
% SOCIAL ENGINEERING & AI
% --------------------------------
    

\chapter{Social Engineering and AI\label{chapter:definition}}
\begin{comment}

Guides:
    - Page limit 1-2 pages
    - Context and terminology (käsitteet), challenges and measurement criteria, values, research question analysis
    - Second to last paragraph contains the research question (RQ) and the results

TODO:
    [ ] 

What to cover:
    - What is cybersecurity and why it's of paramount importance
    - What is social engineering
        - Brief history of social engineering
            - Phishing in 1996 via AOL
    - Attacks, classical social engineering attacks
        - Phishing, vishing, smishing
        - Tailgating
        - Baiting (not always considered SE)
        - Dumpster diving (not always considered SE)
    - Countermeasures, classical
        - User awareness & training programs
        - Company policy & company culture
        - Real-time threat detection
        - Vulnerability detection
    - Typical challenges
    - Motives for cybercrimes
        - Hard(er) to detect?
        - "Easy" wins?

Literature:
    - Defining Social Engineering in Cybersecurity

\end{comment}

This chapter gives an overview of what social engineering constitutes, provides brief historical context and describes some key terminology that is necessary for further analysis, including about AI. After this, Chapter \ref{chapter:attacks}  examines AI-powered attack methods and tools.

The term \textit{social engineering} dates back to 1842, when it was used to describe centralized planning in an attempt to manage the future development and behavior of a society \citep{hatfieldSocialEngineeringCybersecurity2018a}. Since then, its use has shifted to the field of cybersecurity through the phone phreaking phase (late 1950s to early 1970s) and through to the contemporary hacker culture \citep{wangDefiningSocialEngineering2020}.

As one of the earliest hackers and social engineers, the phreakers, used impersonation to call the Bell Telephone company in order to gain insider information about the telephone networks in order to carry out further attacks without the need for social manipulation \citep{hatfieldSocialEngineeringCybersecurity2018a}, modern hackers view social engineering not as something to be replaced but a key part of any hacker's toolkit, in fact perhaps the most important one \citep{mitnickArtDeceptionControlling2003}.

%Interconnected computers on a network are usually represented via a node structure. Business operations, from the perspective of social engineering attacks, can also be represented in this way, with the addition of other attack vectors into the map, such as employees, contractors, entryways, trash dumpsters and any and all other means an attacker could attack an organization. Thus the organization forms a system, with the computers and other network equipment forming only a part of the system, with at least one person always using it. Figure 3 represents such an attacker's view on an organization.

A strict consensus regarding the definition of a social engineering attack is lacking in the field  \citep{hatfieldSocialEngineeringCybersecurity2018a}. For the purposes of this thesis, social engineering is defined as "\textit{a type of attack wherein the attacker(s) exploit human vulnerabilities by means of social interaction to breach cybersecurity, with or without the use of technical means and technical vulnerabilities}" \citep{wangDefiningSocialEngineering2020}.

Some key concepts, namely open-source intelligence, pretexting, and generative AI, are explained next.









% --------------------------------
% Open-Source Intelligence
% --------------------------------
\section{Open-source intelligence}
\begin{comment}

    - OSINT, sometimes written as OS-INT?
    - Data from publicly available resources
        - Company website
        - Social networking sites
        - Sites like archive.org and Google archives
        - Observing people in real life
    - Does not include calling the company and asking for information or any other forms of engagement
    - How modern AI augments OSINT gathering is analyzed in the last chapter
        - Exploration of how AI tools and techniques used for the automation and enhancement of OSINT processes
    - Stress the importance of OSINT within SE
    - Ethical considerations when it comes to OSINT
    - Some case studies highlighting the use of OSINT in real-world social engineering incidents?
    - Countermeasures will also be covered later
        - Strategies for companies to mitigate the risks associated with OSINT-based attacks
        - Integration of AI algorithms for analyzing and extracting valuable insights from OSINT data
        - Impact of AI-powered intelligence gathering of SE attacks
        
\end{comment}

In social engineering, publicly available information is referred to as \textbf{open-source intelligence}. Like the name implies, it involves gathering of intelligence data from publicly locatable sources, such as from the target company's website, or from the social networking profiles of an individual or from other public records.

Various online tools exist for the purposes of gathering intelligence on an individual or an organization, the most famous of which in 2024 is perhaps Maltego. It offers automated forensic gathering and visualizes the found data, making it easier to identify patterns and connections.

Social engineering attacks typically begin with the gathering of open-source intelligence, which are subsequently used in conjunction with pretexting to attack an individual or an organization.










% --------------------------------
% Pretexting
% --------------------------------

\section{Pretexting}
\begin{comment}

    - General info about what is pretexting
    - Fabricated scenario that is plausible but fraudulent
    - Originally used by FBI
    - Impersonation
    - Discussion about how modern AI can aid with pretexting is in the final chapter
    - Role in the deception-based SE attacks
    - Common pretexting tactics will be covered later
    - How AI powers up pretexting will be discussed later
        - How AI tech can be utilized to create more sophisticated and convincing pretexts
        - Examples of successful pretexting attacks and their impacts
        - AI and automated pretexting attacks and their effectiveness
        - Analysis of pretexting evolving landscape with AI
    - Ethical considerations?
    - Countermeasures will be covered later also
        - How to identify and mitigate attempts
        - Recommendations for organiations to enhance their defenses against pretexting attacks
        
\end{comment}

Pretexting involves fabricating a story or a scenario, a \textbf{pretext}, that is plausible but fraudulent, to engage the target  with \citep{contehCybersecurityRisksVulnerabilities2016, fakhouriAIDrivenSolutionsForSocialEngineeringAttacks2024}. With this story, the attacker hopes to gain the victim's trust by appearing legitimate. This type of attack relies heavily on the gathered open-source intelligence in assisting with the creation of the story \citep{hadnagySocialEngineering2018}.

Pretexting uses psychological manipulation, trust and relationship-building, making it a potent tool for attackers \citep{mitnickArtDeceptionControlling2003}. The attacker, often assuming the likeness and character of a legitimate entity such as a trusted colleague, an IT service worker, a government official, or a 3rd party service provider, creates a believable narrative story tailored to the target victim's context.

%The success of pretexting is based on the attacker's ability to gather background OSINT and use it convincingly, making the pretext appear legitimate and aligned with the target victim's expectations or experiences \citep{mitnickArtDeceptionControlling2003}.














% --------------------------------
% Generative AI
% --------------------------------

\section{Generative AI}
\begin{comment}

Artificial Intelligence, Generative AI (ChatGPT, etc)

What to cover:
    - Mitä tekoäly oikeastaan edes on?
    - What is Generative AI
    - OpenAI releasing ChatGPT to the public in 2022
    - NLP Natural Language Processing
    
What to skip:
    - GPT:n historian (versiot 1, 2, 3, 3.5 jne) eli keskitytään vain GPT versioon 4 ja uudempiiin
    
\end{comment}

When AI is used to generate content, it is called \textbf{generative AI} \citep{goodfellowGenerativeAdversarialNetworks2020}. Unlike traditional AI, which follows programmed rules, generative AI utilizes machine learning to learns patterns from large training datasets to produce new outputs, such as text, images, audio and video \citep{fakhouriAIDrivenSolutionsForSocialEngineeringAttacks2024}.

A key example of generative AI is ChatGPT\footnote{https://openai.com/index/chatgpt (accessed 2024-08-19)}, a chatbot released by OpenAI in 2022. While far from being the first \citep{weizenbaumELIZA1996}, this chatbot revolutionized how people use and interact with AI systems, reaching over 100 million users in just two months\footnote{https://explodingtopics.com/blog/chatgpt-users (accessed 2024-08-11)}. Built on the GPT (Generative Pre-trained Transformer) architecture, ChatGPT is designed to understand and generate human-like text by predicting the next word in a sequence. It utilizes natural language processing (NLP), a domain at the intersection of human language and computation.

Another prominent form of generative AI is OpenAI's DALL-E project. It understands human written text and generates images based on the user's prompt.
