

    %%%%%%%%%%%%%%%%%%%%%%%%%%%%%%%%%%%%%%%%%%%%%%%
    %% INTRODUCTION                              %%
    %%%%%%%%%%%%%%%%%%%%%%%%%%%%%%%%%%%%%%%%%%%%%%%


\chapter{Introduction\label{intro}}
\begin{comment}

Guides:
    - Find more things into this guides list
    - Page limit 1-2 pages
    - Context, who needs, what is needed, why is it a problem
    - Research question and research methodologies
    - Summary of results
    - What roles each section will have

TODO:
    [ ] Who is this thesis for
    [ ] Why should you read my thesis
    [ ] What is the research question and how it is answered
    [ ] How is this thesis organized, what is covered and what is deliberaly not covered and in what chapters (outline)

What to cover:
    - Motivate the user to read my thesis!
    - What is cybersecurity and why it's of paramount importance
    - What is SE?
    - Modern AI
        - Large Language Models
        - Emergence of ChatGPT and the like
    - Attacks
        - Deepfake synthetic content, videos, live voice morphing
        - Highly personalized phishing content (natural language processing)
        - Automated OSINT gathering
    - AI augments both attacks and countermeasures
    - Countermeasures
        - User awareness & training programs
        - Company policy & company culture
        - Real-time threat detection
        - Vulnerability detection
    - Emerging challenges
    - Motives for cybercrimes
        - Hard(er) to detect?
        - "Easy" wins?
    - Evolution of SE tactics and their impact on organizations
    - Intro to AI and its applications in cybersecurity
    - Research ga in the literature regarding the intersection of SE and AI


Info from lecture materials:
    - Johdannon tarkoituksena on kertoa yleiskielisesti työn tavoite. Kerrotaan (kuten tiivistelmässäkin, mutta laveammin), mitä on tutkittu, miten on tutkittu ja mitä tuloksia on saatu. Jotta kysymyksenasettelu ja tulokset on lukijan helppo oikein tulkita on syytä aloittaa johdanto asettelemalla tutkimus asiayhteyteensä, esimerkiksi kertomalla aluksi, minkälaisessa yhteydessä tarkasteluun otettavat haasteet esiintyvät ja keiden on ratkaisuista tarkoitus hyötyä.

    - Johdannon pituus määräytyy suhteessa koko kirjoitelman pituuteen. Parisivuinen kirjoitus ei erikseen otsikoitua johdantoa kaipaa, sillä se itsessään on laajennettu tiivistelmä. Kymmensivuisen kirjoituksen johdanto voi olla vaikkapa sivun tai puolentoista mittainen. Pro gradu -tutkielman 50-70-sivuiseen kokonaisuuteen tuntuu 2-4-sivuinen johdanto kohtuulliselta.

    - Johdanto kertoo siis lyhyessä, yleistajuisessa muodossa koko kirjoitelman kysymyksenasettelun, juonen sekä tulokset ja johtopäätelmät. Tämän luettuaan lukija voi päätellä, haluaako syventyä asiaan tarkemmin lukemalla koko kirjoituksen.

\end{comment}

The widespread adoption of information technology (IT) technologies and services has transformed every aspect of human life, from personal communication to business operations, and this reliance on devices and technologies is ever expanding. Although this digital revolution has opened up many opportunities, it has also brought about considerable vulnerabilities. One of the most dangerous threats to security and privacy is social engineering.

Social engineering (SE) is the art and science of manipulating people into performing actions or divulging confidential information. Rather than looking for technical vulnerabilities, SE relies on human interaction and exploits weaknesses in human psychology \citep{wangDefiningSocialEngineering2020}.

With the advent of modern artificial intelligence (AI), the landscape of social engineering is undergoing significant transformation, augmenting the sophistication and effectiveness of such attacks. Based on an in-depth literature review, this thesis explores the intersection of AI and SE, focusing on how contemporary AI technologies enhance the execution and impact of SE attacks and discusses the necessary actions to counter such advanced attacks. For illustration, case studies are also examined. Losses incurred by organizations in sophisticated social engineering attacks have topped \$100 million\footnote{https://www.justice.gov/usao-sdny/pr/lithuanian-man-pleads-guilty-wire-fraud-theft-over-100-million-fraudulent-business}.

Historically, SE relied heavily on human intuition and manual effort to deceive targets. Today, AI systems are more and more capable of automating and amplifying these deceptive practices, often with alarming precision. From sophisticated phishing schemes utilizing natural language processing to deepfake technologies creating convincing fake identities, AI-powered attacks represent a profound shift in the cybersecurity threat landscape.

Certain social engineering attacks that are of particular interest when it comes to advanced AI were chosen for more in-depth analysis, such as phishing, vishing (voice phishing), automated intelligence gathering, and deepfake-generated content, while leaving other attacks with less focus, such as dumpster diving, shoulder surfing, and tailgating.

Countermeasures that are especially covered are AI-based detection systems and how current cybersecurity training programs need to be augmented for the modern threat of AI.

%Next, key ideas and concepts in the field of SE are examined before moving on to the anatomy of a typical SE attack so that the actual attacks can be discussed in their proper context, all of this before we can finally delve into the countermeasures against the discsused attacks.

This thesis is organized into five chapters, each focused on a specific aspect of the interplay between AI and SE. The first chapters build the groundwork for further analysis, beginning with Chapter 1 which provides an overview of social engineering, including its definition and some background knowledge.

Chapter 2 delves into the various stages of a typical social engineering attack, highlighting the tactics and strategies employed by attackers. Chapter 3 examines common attack methods, such as phishing, vishing, pretexting, and tailgating, and explores how modern AI capabilities amplify their effectiveness.

Finally, with the groundwork being built, Chapter 4 discusses AI-powered countermeasures and their potential to detect and prevent social engineering attacks along with other necessary steps to counter AI attacks. Finally, Chapter 5 concludes the thesis, summarizing the key findings and implications for the future of social engineering defense.



%Despite advancements in cybersecurity measures, SE remains a persistent threat. High-profile data breaches and corporate espionage incidents underscore the efficacy of social engineering techniques. Often, sophisticated firewalls, encryption algorithms and intrusion prevention systems (IPS) and intrusion detection systems (IDS) can be rendered useless by a cleverly worded email or an unsuspecting phone call. While it's often said that the most secure computer is one that is not plugged in, with SE an attacker could persuade someone to plug it in and turn it on.

%This study seeks to investigate the diverse aspects of SE, focusing on its techniques, effects, and prevention measures. Grasping the concept of SE is essential, not just for cybersecurity experts but also for people and organizations in every sector and at all employment levels. The increasing sophistication of social engineering attakcks coupled with the integration of artificial intelligence (AI) and technologies such as deepfakes (deep learning, fake) pose a growing risk that transcends traditional security frameworks. Because of this, extra attention is paid to how AI is and might contribute to the field of SE, both in terms of attacks and defenses.

%By dissecting real-world case studies and theoretical underpinnings, this thesis will provide a comprehensive overview of SE, contributing to the broader effort of mitigating its risks.

%Efficacy of SE training and awareness programs varies, with main challenges lying in user's interest and motivation towards the training and retention of learned data. Continued education, especially because SE is a constantly evolving field where the attackers never cease to improvise, is absolutely essential. Gamification, and users's inward motivation, go a long way in achievehing this.



%Usually when IT experts talk about an information system, they focus on the technical parts, the devices. But most, if not all, IT systems are useless without at least some interface with the people using it. Thus, especially from the SE standpoint, it's more beneficial to think of an information system as comprising the technology and its users. With this in mind, it becomes evident that the attack surface a hacker has is vast, and like water flowing down a river, the hacker often chooses the path of least resistance. Time and time again it has been proven that the easiest, most cost-effective way in is through the human factor of the IT system. Bolstering just one facet of this system is like fixing a leak or a potential leak only in some places, and not looking at the whole situation holistically. Security and privacy are a 360 degree, 24/7 thing.

%Currently, social engineering attacks are the biggest threat facing cybersecurity.


%This thesis is organized as follows. First, social engineering is defined and some history explained. Next, the different stages of a social engineering attack are examined. Then multiple common attack methods are discussed along with how modern AI powers them up. After that, AI-powered countermeasures are examined. The final chapter conclues this thesis.



