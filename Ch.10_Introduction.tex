
\chapter{Introduction\label{intro}}

Cybersecurity is

Social engineering is

In this article, we'll go through

How modern AI is

Finally, we'll conclude with some speculation about future AI-based attacks.

\citep{wang_defining_2020,latexcompanion,knuth99}
\cite{wang_defining_2020,latexcompanion,knuth99}

%% Johdannon tarkoituksena on kertoa yleiskielisesti työn tavoite. Kerrotaan (kuten tiivis￾telmässäkin, mutta laveammin), mitä on tutkittu, miten on tutkittu ja mitä tuloksia on saatu. Jotta kysymyksenasettelu ja tulokset on lukijan helppo oikein tulkita on syytä aloittaa johdanto asettelemalla tutkimus asiayhteyteensä, esimerkiksi kertomalla aluksi, minkälaisessa yhteydessä tarkasteluun otettavat haasteet esiintyvät ja keiden on ratkaisu￾ista tarkoitus hyötyä.

%% Johdannon pituus määräytyy suhteessa koko kirjoitelman pituuteen. Parisivuinen kirjoi￾tus ei erikseen otsikoitua johdantoa kaipaa, sillä se itsessään on laajennettu tiivistelmä. Kymmensivuisen kirjoituksen johdanto voi olla vaikkapa sivun tai puolentoista mittainen. Pro gradu -tutkielman 50-70-sivuiseen kokonaisuuteen tuntuu 2-4-sivuinen johdanto ko￾htuulliselta.

%% Johdanto kertoo siis lyhyessä, yleistajuisessa muodossa koko kirjoitelman kysymyksenaset￾telun, juonen sekä tulokset ja johtopäätelmät. Tämän luettuaan lukija voi päätellä, halu￾aako syventyä asiaan tarkemmin lukemalla koko kirjoituksen.