

    %%%%%%%%%%%%%%%%%%%%%%%%%%%%%%%%%%%%%%%%%%%%%%%
    %% INTRODUCTION                              %%
    %%%%%%%%%%%%%%%%%%%%%%%%%%%%%%%%%%%%%%%%%%%%%%%


\chapter{Introduction\label{intro}}

The widespread adoption of information technology (IT) technologies and services has transformed every aspect of human life, from personal communication to business operations, and this reliance on devices and technologies is ever expanding. Although this digital revolution has opened up many opportunities, it has also brought about considerable vulnerabilities. One of the most dangerous threats to security and privacy is social engineering. Social engineering (SE) is the art of manipulating people into performing actions or divulging confidential information. Rather than looking for technical vulnerabilities, SE relies on human interaction and exploits weaknesses in human psychology \citep{wang_defining_2020}.

%Social engineering, however, is not a novel concept. It has been employed in various guises throughout human history. However, in the context of modern computer science, it refers speficially to the manipulation of individuals into performing actions or divulging confidential information. Unlike concentional cyberattacks that exploit software and hardware weaknesses, SE preys on human fallibility. The ability of an attacker to exploit trust, curiosity, fear, greed, ignorance or fatigue can yield significant access to sensitive information, often with minimal risk of detection.

%Despite advancements in cybersecurity measures, SE remains a persistent threat. High-profile data breaches and corporate espionage incidents underscore the efficacy of social engineering techniques. Often, sophisticated firewalls, encryption algorithms and intrusion prevention systems (IPS) and intrusion detection systems (IDS) can be rendered useless by a cleverly worded email or an unsuspecting phone call. While it's often said that the most secure computer is one that is not plugged in, with SE an attacker could persuade someone to plug it in and turn it on.

%This study seeks to investigate the diverse aspects of SE, focusing on its techniques, effects, and prevention measures. Grasping the concept of SE is essential, not just for cybersecurity experts but also for people and organizations in every sector and at all employment levels. The increasing sophistication of social engineering attakcks coupled with the integration of artificial intelligence (AI) and technologies such as deepfakes (deep learning, fake) pose a growing risk that transcends traditional security frameworks. Because of this, extra attention is paid to how AI is and might contribute to the field of SE, both in terms of attacks and defenses.

%By dissecting real-world case studies and theoretical underpinnings, this thesis will provide a comprehensive overview of SE, contributing to the broader effort of mitigating its risks.

%Efficacy of SE training and awareness programs varies, with main challenges lying in user's interest and motivation towards the training and retention of learned data. Continued education, especially because SE is a constantly evolving field where the attackers never cease to improvise, is absolutely essential. Gamification, and users's inward motivation, go a long way in achievehing this.



%Usually when IT experts talk about an information system, they focus on the technical parts, the devices. But most, if not all, IT systems are useless without at least some interface with the people using it. Thus, especially from the SE standpoint, it's more beneficial to think of an information system as comprising the technology and its users. With this in mind, it becomes evident that the attack surface a hacker has is vast, and like water flowing down a river, the hacker often chooses the path of least resistance. Time and time again it has been proven that the easiest, most cost-effective way in is through the human factor of the IT system. Bolstering just one facet of this system is like fixing a leak or a potential leak only in some places, and not looking at the whole situation holistically. Security and privacy are a 360 degree, 24/7 thing.

%Currently, social engineering attacks are the biggest threat facing cybersecurity.

%Hopefully, the readers of this thesis will less likely to be a victims of an SE attacker.

%This thesis is organized as follows. First, social engineering is defined and some history explained. Next, the different stages of a social engineering attack are examined. Then multiple common attack methods are discussed along with how modern AI powers them up. After that, AI-powered countermeasures are examined. The final chapter conclues this thesis.

This thesis is organized into five chapters, each focusing on a specific aspect of social engineering and AI-augmented attacks. Chapter 1 provides an overview of social engineering, including its definition and historical context. Chapter 2 delves into the various stages of a social engineering attack, highlighting the tactics and strategies employed by attackers. Chapter 3 examines common attack methods, such as pretexting, phishing, and tailgating, and explores how modern artificial intelligence (AI) capabilities amplify their effectiveness. Chapter 4 discusses AI-powered countermeasures and their potential to detect and prevent social engineering attacks. Finally, Chapter 5 concludes the thesis, summarizing the key findings and implications for the future of social engineering defense.

% siirrä puhelimella kirjoitettuja tekstejä docsista tänne! pidä kaikki teksti täällä??


% motivoi lukijaa miksi tämä kandi kannattaisi lukea

% mitä motiiveja verkkorikollisuuteen liittyy

% Johdannon tarkoituksena on kertoa yleiskielisesti työn tavoite. Kerrotaan (kuten tiivis￾telmässäkin, mutta laveammin), mitä on tutkittu, miten on tutkittu ja mitä tuloksia on saatu. Jotta kysymyksenasettelu ja tulokset on lukijan helppo oikein tulkita on syytä aloittaa johdanto asettelemalla tutkimus asiayhteyteensä, esimerkiksi kertomalla aluksi, minkälaisessa yhteydessä tarkasteluun otettavat haasteet esiintyvät ja keiden on ratkaisu￾ista tarkoitus hyötyä.

% Johdannon pituus määräytyy suhteessa koko kirjoitelman pituuteen. Parisivuinen kirjoi￾tus ei erikseen otsikoitua johdantoa kaipaa, sillä se itsessään on laajennettu tiivistelmä. Kymmensivuisen kirjoituksen johdanto voi olla vaikkapa sivun tai puolentoista mittainen. Pro gradu -tutkielman 50-70-sivuiseen kokonaisuuteen tuntuu 2-4-sivuinen johdanto ko￾htuulliselta.

% Johdanto kertoo siis lyhyessä, yleistajuisessa muodossa koko kirjoitelman kysymyksenaset￾telun, juonen sekä tulokset ja johtopäätelmät. Tämän luettuaan lukija voi päätellä, halu￾aako syventyä asiaan tarkemmin lukemalla koko kirjoituksen.

% bloomin taksonomia, analyysia. tulosta taksonomia