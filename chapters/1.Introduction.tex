

\chapter{Introduction\label{chapter:intro}}
\begin{comment}
- Independent of platform, age of equipment, software, antivirus, firewall (defining social engineering)
- pääasia päälauseeseen tai paragrafin ensimmäiseen virkkeeseen
- vältä liian pitkiä ja monimutkaisia rakenteita, mutta tuo selkeästi esiin asioiden väliset syyt ja seuraukset
- lauseenvastikkeiden käyttö vain niille sopivissa paikoissa
- havainnoillisesti käyttäen konkretisointeja, esimerkkejä, case-study tutkimuksia, kuvia, taulukoita
- 

\end{comment}


%
% Social engineering as a threat to cybersecurity
%
In the digital age, social engineering has emerged as a significant threat, impacting individuals and organizations worldwide. As a subdomain of cybersecurity, social engineering is the art and science of manipulating people into revealing confidential information or performing actions that may or may not be in their best interests~\citep{hadnagy_Social_Engineering_The_Science_2018}. Rather than looking for technical vulnerabilities, social engineering relies on human interaction and exploits weaknesses in human psychology~\citep{wang_Defining_Social_Engineering_2020}.





%
% Social engineering before and now
%
Traditionally, social engineering depended heavily on human intuition and manual effort to deceive its targets~\citep{mitnick_The_Art_of_Deception_2003, mirsky_Threat_Offensive_AI_Organizations_2023}. However, with the advent of generative artificial intelligence (AI), the landscape of social engineering is undergoing a significant transformation, augmenting the sophistication and effectiveness of current and emerging attack methods~\citep{fakhouri_AI_Driven_Solutions_SE_Attacks_2024, ibm_Cost_Data_Breach_Report_2024, verizon_Data_Breach_Investigations_Report_2024}. Experts from both industry and academia have unanimously ranked impersonation via deepfake media forgeries as the most significant threat among 32 distinct AI capabilities that can be used against organizations~\citep{mirsky_Threat_Offensive_AI_Organizations_2023}.







%
% What this thesis addresses
%
This thesis addresses how contemporary social engineering defensive countermeasures need to be updated for the novel threat of generative AI in an organizational environment, to minimize annual cybersecurity-related costs. To that end, this thesis examines the intersection of generative AI and social engineering based on published literature and incident examples, detailing how advanced AI tools amplify the execution and impact of these attacks while discussing and evaluating the necessary countermeasures.





%
%What attack vectors and tools are analyzed
%
Relevant social engineering attack vectors and tools are examined, including spear phishing with the help of chatbots like ChatGPT, and impersonation using deepfake-generated content. Countermeasures discussed include AI-powered detection of spear phishing and deepfakes, employee training programs, necessary modifications to organizational cybersecurity policies, and restrictions on AI use. Fully automated social engineering is still at a somewhat theoretical level and was thus excluded from this thesis~\citep{hatfield_SE_Evolution_Concept_2018}.

\newpage

%
% Countermeasures are insufficient
%
Contemporary countermeasures against social engineering attacks are ill-equipped to deal with the sophistication of AI-powered threats~\citep{blauth_AI_Crime_Overview_Malicious_Use_Abuse_2022, king_AI_Crime_Interdisciplinary_Analysis_2019}. Cybersecurity professionals must thus urgently update their tools and strategies, and AI can play a valuable role in this defensive effort as well~\citep{fakhouri_AI_Driven_Solutions_SE_Attacks_2024, tsinganos_Towards_Automated_Recognition_Chat_SE_Enterprise_2018}.





%
% How is this thesis organized
%
The rest of the thesis is structured as follows: Chapter~\ref{chapter:background} introduces social engineering, generative AI, and other essential concepts for further analysis. Chapter~\ref{chapter:attacks} covers relevant attack vectors and tools, including spear phishing and impersonation with deepfakes. Chapter~\ref{chapter:countermeasures} analyzes both technology- and user-oriented countermeasures against these attacks. The effectiveness and viability of these measures are assessed in Chapter~\ref{chapter:discussion}. Chapter~\ref{chapter:conclusions} summarizes key findings and implications for the future of organizational social engineering defense.
