
\begin{otherlanguage}{finnish}
\chapter*{Tekoälyavusteinen käyttäjän\\manipulointi\label{chapter:finnish}}
\begin{comment}
\end{comment}


Tässä kandidaatintutkielman suomenkielisessä lyhennelmässä esitellään tärkeimmät tekoälyavustetut käyttäjän manipulointihyökkäykset sekä puolustuskeinot niihin. Käyttäjän manipuloinnilla (\textit{social engineering}) tarkoitetaan tietoturvan yhteydessä tietojärjestelmän loppukäyttäjään eli ihmiseen kohdistuvaa tietoturvahyökkäystä~\citep{hatfield_SE_Evolution_Concept_2018}. Sen sijaan, että hyökkääjät etsisivät tietojärjestelmistä teknisiä haavoittuvuuksia, he kohdistavatkin hyökkäykset käyttäjään käyttäen hyväksi psykologisia menetelmiä~\citep{wang_Defining_Social_Engineering_2020}.

Historiallisesti käyttäjän manipulointi on ollut riippuvainen ihmisen intuitiosta ja manuaalisesta työstä, mutta nyt moderni tekoäly (\textit{artificial intelligence, AI}) on muuttamassa kenttää~\citep{blauth_AI_Crime_Overview_Malicious_Use_Abuse_2022, king_AI_Crime_Interdisciplinary_Analysis_2019, mirsky_Threat_Offensive_AI_Organizations_2023}. Tekoälyn avulla hyökkääjät pystyvät luomaan erittäin uskottavia ja uhrille kohdennettuja tietojenkalasteluviestejä (\textit{spear phishing}) sekä imitoimaan virallisia tahoja ja toimijoita totuudenmukaisten syväväärennösten (\textit{deepfake}), kuten kuvien, äänen ja jopa videoiden, avulla~\citep{mirsky_Creation_Detection_Deepfakes_2021}.

Yritykset kohtaavat tietoturvauhkia monilta eri tahoilta, kuten esimerkiksi hakkereilta, närkästyneiltä työntekijöiltä ja kilpailijoilta~\citep{mirsky_Threat_Offensive_AI_Organizations_2023}. Tietomurto voi johtaa yrityksen maineen kärsimiseen, asiakkaiden menetyksiin, tuotannollisiin tappioihin, sekä rahallisiin menetyksiin.

Taloudelliset tappiot
Kaikki yritykset eivät julkaise tietoja tietomurroista
Tietomurtojen kustannukset, erityisesti vuositasolla
Yritykset voivat arvioida uusien ohjelmistojen, työntekijöiden koulutuksen, yrityskulttuurien muutoksien ja tietoturvaohjelmistojensa tuomaa hyötyä tarkastelemalla tietomurtojen kustannuksia, joko neljännesvuosittain tai vuositasolla
Jonkinlaista osviittaa tietomurtojen määristä voidaan saada raporteista kuten IBM ja FBI


Kuutointi. Kirjoita muutama minuutti per osa
Kuvaillen
Vertaillen
Yhdistellen (lähi-ilmiöön)
Soveltaen
Analysoiden
Valiten hyviä ja huonoja puolia



\section*{Hyökkäykset ja työkalut}

Tunnetuin käyttäjän manipulointihyökkäys on tietojenkalastelu. Tietojenkalastelu on petollista toimintaa jota tehdään useimmiten sähköpostin tai tekstiviestien välityksellä, jossa hyökkääjä esiintyy luotettavana tahona tavoitteenaan saada uhrilta luottamuksellisia tietoja, kuten salasanan tai luottokortin numeron. Kohdennettu tietojenkalastelu on varta-vasten kohdistettu tiettyyn käyttäjään tai yritykseen sisältäen jotain käyttäjälle olennaista tietoa, kuten hänen roolinsa yrityksessä tai hänen työkavereidensa nimiä~\citep{wang_Defining_Social_Engineering_2020}.

OpenAI julkaisi vuonna 2022 ChatGPT:n, joka mullisti tavan, jolla ihmiset käyttävät tekoälypalveluita. Se keräsi yli 100 miljoonaa käyttäjää ensimmäisen kahden kuukauden aikana\footnote{https://explodingtopics.com/blog/chatgpt-users (vierailtu 2024-07-21)}. ChatGPT on ns. generatiivinen tekoäly (\textit{generative AI}), joka on koulutettu suurella määrällä tietoa koneoppimisen alalajina tunnetuilla hermoverkoilla (\textit{neural networks}) ja joka pystyy tämän pohjalta luomaan uutta vastaavanlaista sisältöä, kuten tekstiä tai kuvia~\citep{fakhouri_AI_Driven_Solutions_SE_Attacks_2024}.

OpenAI ja muut tekoälypalveluita varmistavat yritykset ovat asettaneet käyttöehtoja- ja rajoituksia, joiden puitteissa palvelun käyttö on sallittua ja mahdollista. Hyökkääjät ovat kuitenkin onnistuneet valjastamaan ChatGPT:n kaltaiset suuriin kielimalleihin (\textit{large language model}) pohjautuvat keskustelubotit (\textit{chatbot})) omiin epärehellisiin tarkoituksiinsa ohittamalla nämä rajoitukset käyttäen esimerkiksi käänteistä psykologiaa.

ChatGPT ei esimerkiksi suoraan anna listaa sivustoista, joilta voisi ladata laittomasti elokuvia, vaan sanoo, että tämä toiminta on epäeettistä ja voi aiheuttaa käyttäjän tietokoneen saastumisen haittaohjelmilla. Tällaiset rajoitukset on pystytty ohittamaan useilla eri keinoilla, esimerkiksi sanomalla, että suojellakseen käyttäjää haittaohjelmilta ChatGPT:n pitäisi kertoa sivustoista, joille käyttäjän ei tule mennä. Näin käyttäjä saa haluamansa tiedot käänteisen psykologian avulla~\citep{gupta_From_ChatGPT_to_ThreatGPT_2023}.

Näin hyökkääjät ovat pystyneet käyttämään suurten kielimallien tekoälytyökaluja tietojenkalasteluviestien laatimisessa, mikä on huomattavasti parantanut niiden uskottavuutta.

Syväväärennökset ovat aidolta vaikuttavaa mediasisältöä, kuten kuvia, ääntä tai videoita, jotka on luotu generatiivisen tekoälyn avulla~\citep{goodfellow_Generative_Adversarial_Networks_2020}. Syväväärennöksiä voidaan käyttää esimerkiksi opetusmateriaalina, mutta niitä voidaan käyttää myös petollisiin tarkoituksiin. Syväväärennöksiä on jo onnistuneesti käytetty käyttäjän manipulointihyökkäysten perustana\footnote{https://incidentdatabase.ai/cite/634 (vierailtu 2024-08-24)}.

\section*{Puolustuskeinot}

Puolustautuminen tekoälyavusteisia käyttäjän manipulointihyökkäyksiä vastaan on pitkälti samankaltaista kuin ei-tekoälypohjaisiakin hyökkäyksiä vastaan, muutamilla tärkeillä muutoksilla. Puolustautumiskeinot voidaan karkeasti jakaa tekniikka- ja käyttäjälähtöisiin~\citep{tsinganos_Towards_Automated_Recognition_Chat_SE_Enterprise_2018}. Tekniikkalähtöiset menetelmät käydään läpi ensin sillä käyttäjälähtöiset keinot osin nojautuvat niihin.

Perinteinen tapa suojata käyttäjää tietojenkalasteluviesteiltä on esimerkiksi sähköpostiviestien sääntöpohjainen suodattaminen (\textit{rule-based filtering})~\citep{mirsky_Threat_Offensive_AI_Organizations_2023}. Yksinkertaistettuna se tarkoittaa joukkoa loogisia sääntöjä, joita seuraamalla voidaan jollakin todennäköisyydellä päätellä, onko viesti tietojenkalasteluviesti vai ei. Sääntöpohjainen suodattaminen ei kuitenkaan toimi kovin hyvin tekoälyavusteista tietojenkalastelua vastaan~\citep{fakhouri_AI_Driven_Solutions_SE_Attacks_2024}.

Historiallisesti ei ole ollut tarvetta tarkistaa saatujen kuvien tai videoiden aitoutta, mutta nyt syväväärennösten aikakautena käyttäjä ei voi luottaa näkemänsä materiaalin aitouteen, vaan lisävarmistuksia on tehtävä~\citep{mirsky_Creation_Detection_Deepfakes_2021}. Yksi tapa on käyttää tekoälypohjaisia palveluita syväväärennösten tunnistamiseen, samaan tapaan kuin sähköpostiviestienkin tarkistamiseen.

Käyttäjälähtöiset tavat ovat tyypillisesti olleet käyttäjien kouluttaminen, simuloidut käyttäjän manipulointihyökkäykset, yrityksen tietoturva- ja tietosuojaohjeistusten laatiminen ja käytön valvonta, sekä tietoturva- ja tietosuojatietoisen yrityskulttuurin rakentaminen~\citep{tsinganos_Towards_Automated_Recognition_Chat_SE_Enterprise_2018}.

Tekoälypohjainen käyttäjän manipulointi tuo joitakin muutoksia käyttäjälähtöisille puolustuskeinoille. Ensinnäkin käyttäjät eivät enää voi luottaa siihen, että hyvinkään kirjoitettu viesti ei olisi tietojenkalasteluviesti~\citep{gupta_From_ChatGPT_to_ThreatGPT_2023}. Toiseksi kaikki saatu materiaali, kuten kuvat, äänitiedostot ja videot, saattavat olla syväväärennöksiä, vaikka käyttäjä ei itse pystyisi huomaamaan niissä mitään epätavallista~\citep{blauth_AI_Crime_Overview_Malicious_Use_Abuse_2022}.

Koska tekoälyohjelmat pystyvät automaattisesti etsimään Internetistä tietoa, jota voisi käyttää osana käyttäjän manipulointihyökkäyksiä, myöskään viestit, joissa on maininta joistain käyttäjälle oleellista, ehkä jopa henkilökohtaisista asioista, ei voida enää varmuudella sanoa olevan aitoja.

\section*{Puolustuskeinojen arviointia}

Tässä luvussa arvioidaan puolustuskeinojen tehokkuutta.

Tekoälyavusteisten hyökkäysten torjuminen pohjautuu pitkälti jo käytössä oleville tekniikoille: sisääntulevan viestinnän tarkistaminen, käyttäjien kouluttaminen, simuloidut hyökkäykset, yrityskulttuurin rakentaminen ja tietoturvaohjeistusten ylläpito. Jokaiseen näihin kuitenkin on tehtävä muutoksia generatiivisen tekoälyn luoman uuden uhan vuoksi.


\section*{Yhteenveto}

Vaikuttaa siis siltä, että voimme olettaa tekoälyjärjestelmien nopean kehittymisen jatkuvan, tietoturvauhkien kehittymisen niiden mukana sekä tarpeen jatkuvalle käyttäjien kouluttamiselle ja uusien puolustuskeinojen löytämiselle kasvavan.





\end{otherlanguage}


