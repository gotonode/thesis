\chapter{Attack vectors and tools\label{chapter:attacks}}
\begin{comment}
\end{comment}

This chapter reviews key social engineering attack vectors, the method or pathway that a threat actor uses to gain access to data or resources, and tools relevant to the modern threat of generative AI. It first introduces pretexting and spear phishing, then explains how chatbots like ChatGPT could be manipulated, leading to impersonation attacks with deepfakes and voice calls. After this, Chapter~\ref{chapter:countermeasures} goes over the countermeasures against these attacks.



\section{Pretexting}
\begin{comment}
\end{comment}

%
% Social engineering attacks begin with a pretext
%
Social engineering attacks typically begin with the gathering of open-source intelligence, which is subsequently used in conjunction with pretexting to attack an individual or an organization~\citep{hadnagy_Social_Engineering_The_Science_2018}. Pretexting involves fabricating a story or a scenario, a \textbf{pretext}, that is plausible but fraudulent, to engage the target with~\citep{wang_Defining_Social_Engineering_2020}. With this story, the attacker hopes to gain the victim's trust by appearing legitimate. 

%
% Pretexting uses psychological manipulation etc
%
Pretexting uses psychological manipulation, trust, and relationship building, making it a potent tool for threat actors~\citep{mitnick_The_Art_of_Deception_2003}. The attacker, often assuming the likeness and character of a legitimate entity such as a trusted colleague, an IT service worker, a government official, or a 3rd party service provider, creates a believable narrative story tailored to the target victim's context.

%
% Why humans fall victim to pretexts
%
%Humans possess advanced perceptual and decision-making capabilities shaped by lifelong experiences. Attackers can exploit these mental models by presenting deceptive information via pretexting~\citep{mirsky_Threat_Offensive_AI_Organizations_2023}. Information gathered from target A can potentially be used to pretext target B via techniques as simple as utilizing “insider” information.







\section{Spear phishing and its variants}
\begin{comment}
\end{comment}


%
% What is phishing plus a brief history
%
As the quintessential social engineering attack, \textbf{phishing} is characterized by malicious attempts to gain sensitive information from unaware users, traditionally via email and by using spoofed websites that look like their authentic counterparts~\citep{basit_Comprehensive_Survey_AI_Phishing_Detection_2021}. Phishing has been around since 1996 when cybercriminals began using deceptive emails and websites to steal account information from unsuspecting AOL, or America Online, users~\citep{wang_Defining_Social_Engineering_2020}. When phishing attacks are performed using SMS text messages, it’s called \textbf{smishing}.


%
% Spear phishing and whaling, what are they
%
\textbf{Spear phishing}, on the other hand, is a more targeted version of phishing, where threat actors customize their deceptive messages to a target individual or organization~\citep{fakhouri_AI_Driven_Solutions_SE_Attacks_2024}. Spear phishing that is targeted at high-profile individuals is called \textbf{whaling}.


%
% Spear phishing as a more effective, but also more labor-intensive form of phishing
%
Unlike generic phishing attempts, spear phishing involves gathering detailed information about the victim, via open-source intelligence or otherwise, such as their name, position, and contacts to craft a convincing and personalized message~\citep{wang_Defining_Social_Engineering_2020}. Spear phishing has been shown to be up to four times more successful than generic phishing attempts~\citep{king_AI_Crime_Interdisciplinary_Analysis_2019}. This tailored approach increases the likelihood of the victim falling for the phishing attempt, but has traditionally been a lot more time- and energy-consuming~\citep {mirsky_Threat_Offensive_AI_Organizations_2023}.






\section{Abuse of chatbots like ChatGPT}
\begin{comment}
\end{comment}

%
% Malicious use of chatbots like ChatGPT
%
Malicious actors can use generative AI \textbf{chatbots} such as ChatGPT in their social engineering schemes, but due to the manufacturer's set limits, some workarounds may need to be used~\citep{gupta_From_ChatGPT_to_ThreatGPT_2023}. For instance, when asking ChatGPT to provide links to websites that provide pirated content such as movies results in the chatbot denying the request, stating that downloading pirated content is unethical and may also lead to the user's computer being infected with malware.

%
% Bypassing ethical and behavioral restrictions
%
However, regular users and scholars have found a number of ways to bypass ChatGPT's inherent ethical and behavioral guidelines, such as by using reverse psychology\footnote{https://incidentdatabase.ai/cite/420 (visited on 2024-07-15)}. In the above example, instead of directly asking for links to the pirate websites, the user can say that because they do not want their computer to be infected by malware, ChatGPT should provide links to these sites so that the user can avoid visiting them. This technique has been known to cause ChatGPT to reveal the content the user originally wanted~\citep{gupta_From_ChatGPT_to_ThreatGPT_2023}.

%
% ChatGPT can translate spam messages
%
ChatGPT can effectively translate text from the attacker’s native language to the victim’s, maintaining fidelity and correcting any spelling or grammatical errors. It can even enhance the deceptive message, provided that the models' ethical restrictions have been bypassed successfully~\citep{gupta_From_ChatGPT_to_ThreatGPT_2023}.
Phishing messages have historically been marked by noticeable spelling and grammatical errors~\citep{herley_So_Long_No_Thanks_Externalities_2009}, and people have traditionally been advised to look out for these errors as a hallmark of a phishing message. Increasing the message's fidelity will thus increase the likelihood that the target will fall for the phishing attempt~\citep{blauth_AI_Crime_Overview_Malicious_Use_Abuse_2022}.

%
% ChatGPT can integrate OS-INT data into spam messages
%
Chatbots like ChatGPT can also integrate any gathered intelligence into phishing messages, enhancing their relevance. Additionally, incorporating deepfake content, such as an image or a video of the organization’s CEO issuing demands, can further increase the effectiveness of spear phishing attempts.



\section{Impersonation with deepfakes}
\begin{comment}
\end{comment}

%
% What are deepfakes
%
\textbf{Deepfake}, a portmanteau of "deep learning", a type of machine learning, and "fake", is technology that uses artificial neural networks to create highly convincing fake media, either by altering existing content or creating them from scratch~\citep{mirsky_Creation_Detection_Deepfakes_2021}. When existing content is being altered, it's called reenactment or replacement, and when entirely new content is created, it's called synthesis.

%
% What deepfakes can be; depicting a person saying or doing things
%
Deepfake content can be images, audio, and even full-resolution video~\citep{blauth_AI_Crime_Overview_Malicious_Use_Abuse_2022}. Deepfakes have beneficial use cases, such as the realistic dubbing of foreign films, the re-enactment of historical figures for teaching purposes, video games, and for virtually trying on clothes~\citep{mirsky_Creation_Detection_Deepfakes_2021}. However, these hyper-realistic forgeries can also depict a person saying or doing things that didn't take place, making it increasingly difficult for people and even AI systems to discern what is real and what is fake. 

%
% Deepfakes used for impersonation plus a case example
%
By utilizing deepfakes, threat actors can convincingly impersonate trusted individuals or organizations, enhancing the credibility and even the emotional impact of their deceptive social engineering strategies~\citep{mirsky_Creation_Detection_Deepfakes_2021}. In 2021, complete facial reenactment, such as pose, gaze, blinking, and movements, was achieved with only a minute of training video, suggesting that if a malicious actor wants to reenact an individual, they do not need to gather a lot of video material for this. If video material is not available, threat actors might be able to resort to filming the target person exiting the organization's premises. 


Advanced deepfake technology was famously used in a 2024 incident in a live video conference where the threat actors successfully scammed an organization for~\$25 million\footnote{https://incidentdatabase.ai/cite/634 (visited on 2024-08-24)}.% lisää tähän toinen esimerkki, toinen lause?




\section{Phishing via calls and audio}
\begin{comment}
\end{comment}

%
% What is vishing
%
Phishing that is done using voice is called \textbf{vishing}, from voice phishing~\citep{doan_BTSE_Audio_Deepfake_Detection_2023}. By utilizing traditional telephone systems or VoIP (Voice-over-IP), the attacker calls the victim with a pretext to manipulate them into revealing sensitive information or performing actions that may or may not be in their best interests~\citep{hadnagy_Social_Engineering_The_Science_2018}.


%
% Real-time voice morphing for vishing attacks
%
With real-time voice morphing, a type of deepfake natural speech synthesis, the attacker can effectively and realistically impersonate someone else~\citep{doan_BTSE_Audio_Deepfake_Detection_2023}. This technology converts the attacker's voice, as input, to the chosen person's voice, as output, automatically during the call. It's hard for the human auditory system to distinguish between real and fake voice samples, especially through voice calls which tend to have lower audio fidelity.



%
% Training deepfake models
%
Like all deepfake models, the audio model has to be trained before it can be used~\citep{doan_BTSE_Audio_Deepfake_Detection_2023}. This is done using audio, which can be sourced from places like YouTube, a company website, or by calling the person the attacker wants to mimic the voice of and recording the conversation.



%
% Vishing with deepfakes: a significant concern plus a case example
%
Social engineering with real-time voice morphing of employees' voices has been found to be one of the top threats posed by AI to organizations~\citep{mirsky_Threat_Offensive_AI_Organizations_2023}. The first significant and famous incident occurred back in 2019, where threat actors successfully used deepfake-generated voice during a call to impersonate an authentic entity for monetary gains exceeding 200,000~€\footnote{https://incidentdatabase.ai/cite/200 (visited on 2024-05-13)}.
