\chapter{Conclusions\label{chapter:conclusions}}
\begin{comment}
- Muistuta tutkimuskysymys
- Tärkeimmät tulokset ja perusteet
- Incident costs, numbers = impact
suositukset
- Anakyysia, vertailua, arviointia
- Ei turhia osioita, toistoa
- Puutteita kandissa? Mitä olisi hyvä vielä kertoa?’

\end{comment}

The subfield of social engineering within cybersecurity is undergoing a significant transformation with the advent of generative AI~\citep{fakhouri_AI_Driven_Solutions_SE_Attacks_2024}. This thesis explored how generative AI empowers threat actors in this space and how current countermeasures in an organizational environment need to be updated to reflect this evolving threat landscape.

Generative AI is revolutionizing social engineering attacks, enabling threat actors to use sophisticated tactics like spear phishing~\citep{basit_Comprehensive_Survey_AI_Phishing_Detection_2021}, impersonation with deepfake content~\citep{mirsky_Creation_Detection_Deepfakes_2021} and voice phishing, vishing, with real-time voice morphing~\citep{doan_BTSE_Audio_Deepfake_Detection_2023}. These advancements reveal that traditional countermeasures are becoming increasingly ineffective, requiring an urgent and comprehensive re-evaluation of current strategies and tactics.

Previously an employee could authenticate a caller by recognizing their voice, intonations, and accent~\citep{mitnick_The_Art_of_Deception_2003}, but today this is no longer enough. User training and awareness programs must be updated to address the novel threat of AI in social engineering. Historically, employees have been trained to spot spelling errors in email messages, and today they must be trained to broaden their scope of skepticism to include images, audio, and videos as well~\citep{mirsky_Creation_Detection_Deepfakes_2021}.

AI can help detect social engineering attacks, but it does not eliminate the necessity for user training and awareness programs. On the contrary, as AI-powered attacks proliferate, the need for awareness and vigilance will grow even higher~\citep{fakhouri_AI_Driven_Solutions_SE_Attacks_2024}. Chatbots like ChatGPT can help develop more robust security guidelines and design highly engaging social engineering awareness programs. In addition, image-generation technologies like DALL-E can help create memorable and funny images for posters and campaigns.

One area not addressed in this thesis, but deserving of future research, is the potential for AI to automate social engineering attacks, either in part or even completely~\citep{mirsky_Threat_Offensive_AI_Organizations_2023}. Currently, however, AI technology is not capable of executing such attacks without human oversight, but as the field is evolving rapidly, organizations must take this possibility into consideration as well.


The Cost of a Data Breach Report~\citep{ibm_Cost_Data_Breach_Report_2024} revealed that organizations using AI to address cybersecurity threats experienced an average of 45\% reduction in annual incident-related costs compared to those that did not. Further, IBM found that increased reliance on AI corresponded with lower incident costs. Organizations need to utilize AI to combat generative AI -powered social engineering, primarily because the user is the weakest link in the cybersecurity chain.

%toistoa
% mainitse kustannuksista
What seems certain is that we can count on the rapid development of AI technologies continuing and generative AI -powered social engineering attacks evolving with them. The need for continuous, innovative user training will be growing in the future as well as the need for the development of AI-based mitigation and prevention technologies~\citep{mirsky_Threat_Offensive_AI_Organizations_2023}. Cybersecurity experts must concentrate their efforts on deterring the top threats organizations face from AI, namely social engineering powered by generative AI and impersonation with deepfakes.

It seems evident that the highly dynamic nature of AI technologies fuels a continuous arms race between attackers and defenders, causing many countermeasures to become obsolete quickly~\citep{fakhouri_AI_Driven_Solutions_SE_Attacks_2024}. Thus, protecting against AI-powered attacks requires not a single solution but an integrated approach that is baked in the company culture, that combines technological defenses, comprehensive and continuous user education, and robust organizational policies.

%
% Defense against AI-powered threats
%
Defense against AI-enhanced social engineering will thus require a multifaceted approach that combines technological innovation, user education, and a proactive stance and strict enforcement of cybersecurity policy~\citep{blauth_AI_Crime_Overview_Malicious_Use_Abuse_2022}. As the landscape continues to evolve, staying ahead of these threats will necessitate ongoing research and collaboration across the cybersecurity community to develop effective countermeasures and best practices~\citep{fakhouri_AI_Driven_Solutions_SE_Attacks_2024}.
