

    %%%%%%%%%%%%%%%%%%%%%%%%%%%%%%%%%%%%%%%%%%%%%%%
    %% COUNTERMEASURES                           %%
    %%%%%%%%%%%%%%%%%%%%%%%%%%%%%%%%%%%%%%%%%%%%%%%


\chapter{Countermeasures\label{chapter:countermeasures}}
\begin{comment}


\end{comment}

In this chapter, AI-powered countermeasures against both classic social engineering and AI-powered social engineering are examined. This chapter is divided into two parts: tech-oriented countermeasures such as deepfake detection mechanisms and human-oriented countermeasures such as training and awareness programs. Tech-oriented countermeasures are examined first since the human-oriented measures rely and build upon them. The division is not always clear cut and is made only to simplify the reading experience.

Finally, Chapter \ref{chapter:conclusions} concludes the thesis.

        %%%%%%%%%%%%%%%%%%%%%%%%%%%%%%%%%%%%%%%%%%%%%%%
        %% Technology oriented                       %%
        %%%%%%%%%%%%%%%%%%%%%%%%%%%%%%%%%%%%%%%%%%%%%%%

\section{Technology oriented}
\begin{comment}
    
    - 

\end{comment}

Countermeasures against AI-powered threats to SE that fall into the technology-oriented, versus human-oriented, are discussed.

They include automated phishing email detection powered by AI and machine learning algorithms.


        %%%%%%%%%%%%%%%%%%%%%%%%%%%%%%%%%%%%%%%%%%%%%%%
        %% Human oriented                            %%
        %%%%%%%%%%%%%%%%%%%%%%%%%%%%%%%%%%%%%%%%%%%%%%%


\section{Human oriented}
\begin{comment}
    
    - The best defense against SE attacks is an educated, conscious user
    - User education should be continuous and not a one-off event

\end{comment}

Regular and comprehensive training programs are vital to educate employees about social engineering tactics.

Conducting concensual, simulated phishing attack campaigns, via numerous channels such as email, SMS and even phone/VoIP, allows organizations to assess the suspectibility of their employees to SE tactics. These exercises help identify vulnerabilities in the workforce, enabling targeted training and reinforcing the importance of scrutiziing unsolicited communication. Feedback from these simulations can be a powerful tool for personnel development. It's important that Employees should be informed beforehand that such campaigns may be intermittently run, which has the double benefit of keeping the employees on their guard and also not causing unneccessary bad emotions from "being tricked" by their own company \citep{hadnagySocialEngineering2018, mitnickArtDeceptionControlling2003}.

Regarding bating attacks, company policy should enforce antivirus scanning of attached media, such as USB thumbdrives, before they can be accessed. Personnel should be instructed to never pick up and plug in found media \citep{salahdine_social_2019}. However, if media is found that is suspected of being used as part of a baiting attack, the employee could notify the front desk or the IT services that such a media device was found within the premises, rather than just ignoring it 

Because AI can source the Internet automatically for OSINT, it's imperative for people to know to be careful of what they share, with whom and when. There's a difference between sharing that one is going to a company picnic next week, and posting videos about a picnic that had taken place 3 days prior.



In this chapter, some key countermeasures against pre-AI social engineering as well as AI-powered SE were looked at. The final chapter \ref{conclusions} ties everything together and provides speculation about the current state of the AI SE landscape as well as future prospects.

