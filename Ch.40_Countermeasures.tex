

    %%%%%%%%%%%%%%%%%%%%%%%%%%%%%%%%%%%%%%%%%%%%%%%
    %% TRAINING & AWARENESS PROGRAMS             %%
    %%%%%%%%%%%%%%%%%%%%%%%%%%%%%%%%%%%%%%%%%%%%%%%


\chapter{Countermeasures\label{countermeasures}}
\begin{comment}


\end{comment}


In this chapter, AI-powered countermeasures against both classic social engineering and AI-powered social engineering are examined. This chapter is divided into two parts, human-oriented countermeasures such as training and awareness programs, and tech-orientend countermeasures such as deepfake detection mechanisms. The division is not always clear-cut and is made only to simplify the reading experience.

% mitä käydään läpi
% - käyttäjän koulutus
% - training & awareness programs
% mitä EI käydä läpi
% - ei teknisiä tietoja, deepfake videoiden tunnistus AI:n avulla?
% - ei 

\section{Human-oriented}
\begin{comment}
    
    - 

\end{comment}


Regarding bating attacks, company policy should enforce antivirus scanning of attached media, such as USB thumbdrives, before they can be accessed. Personnel should be instructed to never pick up and plug in found media \citep{salahdine_social_2019}. However, if media is found that is suspected of being used as part of a baiting attack, the employee could notify the front desk or the IT services that such a media device was found within the premises, rather than just ignoring it 

\section{Tech-oriented}
\begin{comment}
    
    - 

\end{comment}
