\chapter{Background\label{chapter:background}}
\begin{comment}
\end{comment}


%
% Generative AI in SE, chapter overview
%
In recent years, the integration of generative artificial intelligence (AI) into social engineering offensive practices has emerged as a significant concern within the field of cybersecurity~\citep{blauth_AI_Crime_Overview_Malicious_Use_Abuse_2022, king_AI_Crime_Interdisciplinary_Analysis_2019, mirsky_Threat_Offensive_AI_Organizations_2023}. This chapter provides an overview of the role of generative AI in social engineering, explaining key concepts and terminologies essential for understanding the evolving threat landscape. After this, Chapter~\ref{chapter:attacks} examines generative AI -powered attack vectors and tools.



%
% Defining social engineering
%
A strict consensus regarding the definition of a social engineering attack is lacking in the field~\citep{hatfield_SE_Evolution_Concept_2018}. For the purposes of this thesis, social engineering is defined as "\textit{a type of attack wherein the attacker(s) exploit human vulnerabilities by means of social interaction to breach cybersecurity, with or without the use of technical means and technical vulnerabilities}"~\citep{wang_Defining_Social_Engineering_2020}.



%
% Cybersecurity threats to organizations
%
Organizations today face cybersecurity threats from a range of sources, including cybercriminals, disgruntled employees, "script kiddies" (amateur hackers), hacktivists, competitors, and even state-sponsored cyber terrorists~\citep{mirsky_Threat_Offensive_AI_Organizations_2023}. These threat actors may be driven by motives such as financial gain, intellectual property theft, sabotage, fame, or revenge. Organizations face public scrutiny, loss of customer trust and relations, governmental fines, and loss of productivity, among other things, due to data breaches.


%
% Threat of generative AI
%
A total of 32 different AI capabilities have been identified that attackers could use against an organization~\citep{mirsky_Threat_Offensive_AI_Organizations_2023}. The top three most threatening categories are (1) social engineering, (2) information gathering, and (3) exploit development. Experts from both academia and industry ranked deepfake-based impersonation as the highest threat~\citep{mirsky_Threat_Offensive_AI_Organizations_2023}. Social engineering attacks are ranked the most threatening because these types of attacks are outside of the defender's control, are relatively easy to achieve, have high payoffs, are hard to prevent, and cause the most harm.



%
% Tracking incidents
%
Tracking social engineering incidents can be accomplished by counting occurrences or by calculating the total cost of all incidents annually~\citep{ibm_Cost_Data_Breach_Report_2024}. Not all organizations report their social engineering and other cybercrime-related incidents, but some estimates of the prevalence of these attacks can be gained from data that is gathered by various public and private organizations and released in reports such as the Internet Crime Report~\citep{fbi_Internet_Crime_Report_2023} and Cost of a Data Breach report~\citep{ibm_Cost_Data_Breach_Report_2024}. Organizations can thus assess the effectiveness and impact of their new policies, software upgrades, and cultural changes by monitoring incident statistics, especially incident-related annual costs.




%
% AI threat 
%
The dynamic nature of AI-driven social engineering poses a significant challenge for traditional cybersecurity frameworks, which often rely on static defenses and predefined patterns of attack~\citep{fakhouri_AI_Driven_Solutions_SE_Attacks_2024}. As generative AI technologies advance, their application in crafting more convincing and personalized social engineering attacks becomes increasingly evident~\citep{blauth_AI_Crime_Overview_Malicious_Use_Abuse_2022}. This new capability not only enhances the likelihood of success but also complicates the detection and mitigation of such threats~\citep{mirsky_Threat_Offensive_AI_Organizations_2023}.



%
% Defense against AI-powered threats
%
Defense against AI-enhanced social engineering will thus require a multifaceted approach that combines technological innovation, user education, and a proactive stance and strict enforcement of cybersecurity policy~\citep{blauth_AI_Crime_Overview_Malicious_Use_Abuse_2022}. As the landscape continues to evolve, staying ahead of these threats will necessitate ongoing research and collaboration across the cybersecurity community to develop effective countermeasures and best practices~\citep{fakhouri_AI_Driven_Solutions_SE_Attacks_2024}.



%
% How this thesis is organized
%
The remainder of this chapter elaborates on essential concepts, specifically open-source intelligence and generative AI, which are essential for further analysis.




\section{Open-source intelligence}
\begin{comment}
Some case studies highlighting the use of OSINT in real-world social engineering incidents?
\end{comment}

Social engineering attacks begin with the gathering of data. In cybersecurity, publicly available information is known as \textbf{open-source intelligence}~\citep{hadnagy_Social_Engineering_The_Science_2018}. This practice involves collecting intelligence from sources that are publicly accessible, such as the target company's website, individuals' social media profiles, or other public records. Attackers are increasingly utilizing platforms such as LinkedIn, Facebook, and X (formerly Twitter) to gather information about their victims ~\citep{fakhouri_AI_Driven_Solutions_SE_Attacks_2024}.

Various online tools have been created for the purposes of gathering intelligence on an individual or an organization~\citep{mirsky_Threat_Offensive_AI_Organizations_2023}. They often offer automated forensic gathering and are able to visualize the found data, making it easier to identify exploitable patterns and connections. Many of these tools are adapting to use powerful AI technologies as well.

Attackers are also able to utilize sites like the Internet Archive and specific web searching features such as Google’s cache to find websites and other material that is no longer accessible via simple web search queries. Bots can be used to download social media posts at frequent intervals in case the target organization makes a mistake in one of their social media posts and deletes it promptly.

Lastly, open-source intelligence, as the name implies, does not contain intelligence gathered using any of the social engineering tactics discussed later, such as calling customer support and asking for personnel information~\citep{hadnagy_Social_Engineering_The_Science_2018}. Open-source intelligence-gathering practices should not leave any traces behind.





\section{Generative AI}
\begin{comment}
\end{comment}

Artificial intelligence (AI) encompasses the development of algorithms designed to automate complex tasks ~\citep{mirsky_Threat_Offensive_AI_Organizations_2023}. Currently, the most prevalent type of AI is machine learning, which enables systems to enhance their performance as they gain experience. Deep learning, a subset of machine learning, employs extensive artificial neural networks as predictive models~\citep{fakhouri_AI_Driven_Solutions_SE_Attacks_2024}. The core idea behind AI is to enable machines to mimic human-like decision-making and thinking processes.

When AI is used to generate content, it is called \textbf{generative AI}~\citep{goodfellow_Generative_Adversarial_Networks_2020}. Unlike traditional AI, which follows programmed rules, generative AI utilizes machine learning to learn patterns from large training datasets to produce new outputs, such as text, images, audio and video.

Perhaps the most prominent example of generative AI is ChatGPT\footnote{https://openai.com/index/chatgpt (visited on 2024-08-19)}, a chatbot released by OpenAI in 2022. While far from being the first~\citep{weizenbaum_ELIZA_1996}, this chatbot revolutionized how people use and interact with generative AI systems, reaching over 100 million users in just two months\footnote{https://explodingtopics.com/blog/chatgpt-users (visited on 2024-08-11)}. Built on the GPT (Generative Pre-trained Transformer) architecture, ChatGPT is designed to understand and generate human-like text by predicting the next word in a sequence.

Another relevant generative AI technology for social engineering is DALL-E\footnote{https://openai.com/index/dall-e-3/ (visited on 2024-09-19)}, also developed by OpenAI. This system generates images from textual descriptions, facilitating digital manipulation and the creation of misleading visuals. It enables the production of hyper-realistic images that can distort or shape public perception.
