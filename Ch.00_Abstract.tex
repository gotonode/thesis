

    %%%%%%%%%%%%%%%%%%%%%%%%%%%%%%%%%%%%%%%%%%%%%%%
    %% ABSTRACT (English)                        %%
    %%%%%%%%%%%%%%%%%%%%%%%%%%%%%%%%%%%%%%%%%%%%%%%


\begin{otherlanguage}{english}
\begin{abstract}

This thesis explores the intersection of artificial intelligence (AI) and social engineering, examining the emerging threats as well as opportunities for defense in this space.

The findings reveal that AI-powered attacks are more convincing and successful than traditional attacks, but AI-driven defense mechanisms can improve detection rates, and AI can develop more engaging training material(?).

% mitä käsitellä
% cybersecurity
% se in cybersecurity
% modern ai developments, llm's
% ai-powered social engineering attacks
% & countermeasures

% The integration of artificial intelligence (AI) into social engineering practices presents exceptional threats to privacy and security, necessitating a truly comprehensive re-evaluation, and enhancement, of current cybersecurity measures.

The study contributtes to our understanding of AI's impact on social engineering and highlights the need for cybersecurity professionals to adapt to these emerging threats.

While AI can help detect social engineering attacks, it does not mitigate the need for user training and awareness programs, quite the contrary, with AI-powered attacks the need for awareness and vigilance will likely grown even higher.

Technologies such as deepfake (deep learning, fake) videos and live voice morphing will change the landscape of SE attacks.

As the field is still emerging and the field of AI development has seen increased growth vastly, much of this the content of this thesis is speculative, based on attacks that have already been acted out succesfully and the developments on the AI field.

What, if anything, can the end-user trust anymore?

\end{abstract}
\end{otherlanguage}

% max 100 sanaa? laske sanat (voiko sitä tehdä Overleafissa)
% jos tiivistelmäsivu tulee liian täyteen, onko se demotivoivaa?
% määrittele SE lyhyesti, mitä uhkia, esimerkkejä
% kerro tekoälyn kehittymisestä
% - suuret kielimallit, ChatGPT jne
% - deepfake videot ja sisältö
% - äänen muuttaminen
% - data mining via AI
% - automaattiset AI hyökkäykset

% Tiivistelmäteksti on lyhyt, yleensä yhden kappaleen mittainen (maksimissaan noin 100 sanaa) selvitys kirjoituksen tärkeimmästä sisällöstä: mitä on tutkittu, miten on tutkittu ja mitä tuloksia on saatu.

