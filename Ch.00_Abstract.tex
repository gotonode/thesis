% \begin{abstract}{finnish}

% Tämä dokumentti on tarkoitettu Helsingin yliopiston tietojenkäsittelytieteen osaston opin\-näyt\-teiden ja harjoitustöiden ulkoasun ohjeeksi ja mallipohjaksi. Ohje soveltuu kanditutkielmiin, ohjelmistotuotantoprojekteihin, seminaareihin ja maisterintutkielmiin. Tämän ohjeen lisäksi on seurattava niitä ohjeita, jotka opastavat valitsemaan kuhunkin osioon tieteellisesti kiinnostavaa, syvällisesti pohdittua sisältöä.


% Työn aihe luokitellaan  
% ACM Computing Classification System (CCS) mukaisesti, 
% ks.\ \url{https://dl.acm.org/ccs}. 
% Käytä muutamaa termipolkua (1--3), jotka alkavat juuritermistä ja joissa polun tarkentuvat luokat erotetaan toisistaan oikealle osoittavalla nuolella.

% modern AI threats: robocalls, fully automated social engineering attacks, personalized phishing emails, amplified OSINT open source intelligence, learning from other social engineering attacks, learn even from psychological research?

% \end{abstract}

\begin{otherlanguage}{english}
\begin{abstract}

The integration of artificial intelligence (AI) into social engineering practices presents exceptional threats to privacy and security, necessitating a truly comprehensive re-evaluation, and enchantment, of current cybersecurity measures. Presenting both novel threats, as well as novel learning opportunities as countermeasures for AI attacks.

Cybersecurity is often broken down into three parts, representing the CIA Triad -model[1], which stands for Confidentiality, Integrity, and Availability. Future needs have necessitated that Auditability and Controllability be added to these. These three parts are the core principles designed to guide information security policies and practices within an organization. Let’s take a look at each of these next.



Some experts say that AI will ruin Internet. SOURCE

What, if anything, can the end user trust anymore?

In addition, make sure that all the entries in this form are completed.

Finally, specify 1--3 ACM Computing Classification System (CCS) topics, as per \url{https://dl.acm.org/ccs}.
Each topic is specified with one path, as shown in the example below, and elements of the path separated with an arrow.
Emphasis of each element individually can be indicated
by the use of bold face for high importance or italics for intermediate
level.

\end{abstract}
\end{otherlanguage}
