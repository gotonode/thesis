

    %%%%%%%%%%%%%%%%%%%%%%%%%%%%%%%%%%%%%%%%%%%%%%%
    %% ABSTRACT (English)                        %%
    %%%%%%%%%%%%%%%%%%%%%%%%%%%%%%%%%%%%%%%%%%%%%%%


\begin{otherlanguage}{english}
\begin{abstract}

\begin{comment}

Guides:
    - Max about 100-200 words
        - Some other theses have almost fully covered the available area
    - If the abstract page is too full, is that demotivating to the reader?
    - Study material states that the abstract text is short, usually just one paragraph

TODO:
    [ ] What is the research question
    [ ] What is social engineering, quick definition?
    [ ] How does modern AI augment SE attacks and countermeasures
    [ ] What is analyzed in this thesis
    [ ] Properly motivate the reader to read my thesis
    [ ] Explain that this is written in an easy-to-read manner?

What to cover:
    - What is SE?
    - Modern AI
    - Attacks
        - Deepfake synthetic content, videos, live voice morphing
        - Highly personalized phishing content (natural language processing)
        - Automated OSINT gathering
    - AI augments both attacks and countermeasures
    - Countermeasures
        - User awareness & training programs
        - Company policy & company culture
        - Real-time threat detection
        - Vulnerability detection
    - Emerging challenges

\end{comment}

This thesis explores the evolving landscape of social engineering (SE) in the age of modern artificial intelligence (AI). While AI offers modern opportunities for enhancing cybersecurity measures, it simultaneously empowers malicious actors with sophisticated tools for crafting highly targeted and effective social engineering attacks.

Delving into the various ways AI is being exploited to augment social engineering tactics, including the creation of highly believable synthetic media like deepfake images, videos and real-time voice morphing, the generation of personalized phishing messages through natural language processing.

Conversely, the thesis also explores how AI can be harnessed to bolster countermeasures by enabling real-time threat detection, identifying potential vulnerabilities and facilitating comprehensive employee training programs. By analyzing the dual-faceted impact of AI on SE, this thesis aims to provide a comprehensive understanding of the emerging challenges and opportunities in this domain.

Given the rapidly evolving nature of AI and its expanding capabilities, much of the content presented is speculative.These projections are grounded in analysis of established social engineering attacks and observed trends in AI development.

%This thesis explores the intersection of artificial intelligence (AI) and social engineering, examining the emerging threats as well as opportunities for defense in this space.

%The findings reveal that AI-powered attacks are more convincing and successful than traditional attacks, but AI-driven defense mechanisms can improve detection rates, and AI can develop more engaging training material(?).

% The integration of artificial intelligence (AI) into social engineering practices presents exceptional threats to privacy and security, necessitating a truly comprehensive re-evaluation, and enhancement, of current cybersecurity measures.

%The study contributtes to our understanding of AI's impact on social engineering and highlights the need for cybersecurity professionals to adapt to these emerging threats.

%While AI can help detect social engineering attacks, it does not mitigate the need for user training and awareness programs, quite the contrary, with AI-powered attacks the need for awareness and vigilance will likely grown even higher.

%Technologies such as deepfake (deep learning, fake) videos and live voice morphing will change the landscape of SE attacks.

% Tiivistelmäteksti on lyhyt, yleensä yhden kappaleen mittainen (maksimissaan noin 100 sanaa) selvitys kirjoituksen tärkeimmästä sisällöstä: mitä on tutkittu, miten on tutkittu ja mitä tuloksia on saatu.




\end{abstract}
\end{otherlanguage}
