

% --------------------------------
% ABSTRACT (English)
% --------------------------------

\begin{otherlanguage}{english}
\begin{abstract}

\begin{comment}

Guides:
    - Max about 100-200 words
        - Some other theses have almost entirely covered the available area
    - If the abstract text is too full, is that demotivating to the reader?
    - Study material states that the abstract text is short, usually just one paragraph?
    - What has been studied
    - How it has been studied
    - What results have been observed

TODO:
    [x] What is the research question
    [x] What is social engineering, quick definition?
    [x] How does modern AI augment SE attacks and countermeasures
    [x] What is analyzed in this thesis
    [x] Properly motivate the reader to read my thesis
    [ ] Explain that this is written in an easy-to-read manner?
    [x] What results have been observed

What to cover:
    - What is SE?
    - Modern AI
    - Attacks
        - Deepfake synthetic content, videos, live voice morphing
        - Highly personalized phishing content (natural language processing)
        - Automated OSINT gathering
    - AI augments both attacks and countermeasures
    - Countermeasures
        - User-oriented
            - User awareness & training programs
            - Company policy & company culture
        - Tech oriented
            - Real-time threat detection
            - Vulnerability detection
    - Emerging challenges

From the student's material:
    - "Tiivistelmäteksti on lyhyt, yleensä yhden kappaleen mittainen (maksimissaan noin 100 sanaa) selvitys kirjoituksen tärkeimmästä sisällöstä: mitä on tutkittu, miten on tutkittu ja mitä tuloksia on saatu."

\end{comment}

Social engineering, a subdomain of cybersecurity, is the art and science of manipulating people into divulging confidential information or taking actions that may or may not be in their best interests. Traditionally, social engineering relied heavily on manual labor and human intuition, but with the advent of modern AI technologies, cybercriminals are able to craft highly targeted and effective social engineering campaigns with novel, unexpected twists.% that are sometimes almost entirely automated.

This thesis explores the evolving landscape of AI in social engineering, focusing on attacks such as spear phishing aided with chatbots like ChatGPT, and impersonation with realistic deepfake-generated content. In contrast, the thesis also covers countermeasures against these attacks and evaluates their effectiveness based on relevant literature. Actualized incidents are examined where appropriate.

The results indicate that AI-powered attacks are more convincing and successful than traditional attacks and that contemporary countermeasures against these attacks are becoming increasingly ineffective. This highlights the urgent need for cybersecurity professionals to update their strategies and tools for cyber defense against the emerging threat of AI, as social engineering is a threat to every organization and every individual.

% professionals to adapt to these emerging threats.

% grounded in an analysis of established social engineering attacks and observed trends in AI development.



\end{abstract}
\end{otherlanguage}
