


\chapter{Conclusions\label{chapter:conclusions}}
\begin{comment}

Guides:
    - 2 pages
    - No subsections

TODO:
    [ ] How AI has augmented SE attacks and countermeasures
    [ ] Gap in the literature

What to cover:
    - How AI has augmented SE attacks and countermeasures
    - Gap in the literature regarding SE and AI intersection?
    - Analysis on where AI-powered SE attacks might be headed in the future
        - Also about robotics and human-like actors
    - What organizations and individuals need to do regarding the evolving landscape of SE attacks

Speculation:
    - Drones dropping USB thumbdrives?
    - Human-like android as threat actors
    - Impact of robotics on dumpster diving, shoulder surfing and baiting
    
From training material:
    - "Yhteenveto vaatimattomimmillaan on vain lyhyt kertaus kirjoituksen keskeisistä asioista. Arvokkaamman yhteenvedon saa aikaan kommentoimalla työn tulosten arvoa, työn liittymistä ympäristöön ja tulevaisuudennäkymiä. Tällaiset arviot huolellisesti perusteltava."
    - "Yhteenvetoluku kuvaa teknisten johtopäätösten tuomaa impaktia."

\end{comment}

The subfield of social engineering within cybersecurity is undergoing a significant transformation with the advent of generative AI \citep{fakhouriAIDrivenSolutionsForSocialEngineeringAttacks2024}. This thesis explored how AI empowers malicious actors and also how current countermeasures need to be updated to reflect this evolving threat landscape.

Modern AI is revolutionizing social engineering attacks, enabling attackers to use sophisticated tactics like spear phishing \citep{basitComprehensiveSurveyAIenabledPhishingAttacks2021}, impersonation with deepfake content \citep{mirskyTheCreationAndDetectionOfDeepfakes2021} and voice phishing (vishing) with real-time voice morphing \citep{doanBTSEAudioDeepfakeDetectiong2023}. These advancements reveal that traditional countermeasures are becoming ever more ineffective, requiring a re-evaluation of current strategies and tactics.

One of the most notable contributions of AI is its ability to automate and enhance deceptive practices. Machine learning facilitates the crafting of personalized phishing messages that closely mimic legitimate communications, while deepfake technologies alter or produce synthetic media that convincingly impersonate authentic images, audio and videos. Such advancements enable attackers to deceive targets more efficiently into disclosing sensitive information or taking actions that compromise security.

AI has potential to increase the scale and reach of social engineering \citep{king_AI_Crime_Interdisciplinary_Analysis_2019}
Dual use of AI, applications that are designed for legitimate use cases may also be implemented to commit criminal offences \citep{king_AI_Crime_Interdisciplinary_Analysis_2019}

Where previously an employee could authenticate a caller by recognizing their voice, intonations, and accents \citep{mitnick_The_Art_of_Deception_2003}, today and especially in the near future this will not be enough. User training and awareness programs need to be updated for novel threat of AI in social engineering.

In their article, \cite{guptaFromChatGPTtoThreatGPT2023}, claim that "\textit{through continued efforts and cooperation among various stakeholders, it’s possible to prevent the misuse of AI systems and ensure their continued benefit to society}", but this can only be true if advanced AI systems remain in the hands of their developers and that they retract older versions of their AI systems from use, since the older versions have already been used by malicious actors. And since with social engineering an attacker can ask ChatGPT to roleplay a certain scenario that the attacker will later enact in a live call, misuse of AI systems can never be fully prevented. AI is a tool, and like any tool it can be used for its intended purpose or in ways the original manufacturer did not intend or would not want.

According to IBM's 2024 Cost of a Data Breach\footnote{https://www.ibm.com/reports/data-breach (accessed 2024-08-11)} report, organizations are increasingly leveraging AI and automation in their security operations, with 2/3's of studied organizations deploying these technologies. This presents a 10\% increase from last year. Notably, when AI was extensively deployed in prevention workflows, including attack surface management (ASM), red-teaming and posture management, these organizations saw an average reduction of \$2.2 million breach costs compared to the average of \$4,88 million, a 45\% reduction. IBM found a striking correlation, that the more an organization relied on AI, the lower their average breach costs were.

AI can help detect social engineering attacks, but it does not eliminate the necessity for user training and awareness programs. On the contrary, as AI-powered attacks proliferate, awareness and vigilance will grow even higher. Chatbots like ChatGPT help develop stronger security guidelines and design more engaging social engineering awareness programs.

Dual use property of AI, software created for defense can also be utilized for offensive purposes \citep{blauth_AI_Crime_Overview_Malicious_Use_Abuse_2022}

The benefits of AI for society and individuals may be significantly compromised due to ongoing constraints on its development \citep{king_AI_Crime_Interdisciplinary_Analysis_2019}. A notable example is the restriction on releasing source code and data from a study that demonstrated how visual discriminators could identify a person's sexual orientation with accuracies far higher than those of human judges, which undermines scientific reproducibility \citep{king_AI_Crime_Interdisciplinary_Analysis_2019}.

Adobe embedded watermarks into their voice reproducing technology, however malevolent developers might still reproduce the technology in the future \citep{king_AI_Crime_Interdisciplinary_Analysis_2019}

AI technologies such as IBM's Watson fo Cyber Security, goes over the organizations organized and unorganized security intel, and things like blog posts, released articles, reports etc with the goal of better threat identification, response and mitigation \citep{king_AI_Crime_Interdisciplinary_Analysis_2019}.

AI social bots can be used to engage millions of users in one-to-one conversations with malicious intents \citep{king_AI_Crime_Interdisciplinary_Analysis_2019}

The faster the potential for AI misuse is understood, the earlier potential preventive, mitigative, disincentivisive and redressing policisi may be applied \citep{king_AI_Crime_Interdisciplinary_Analysis_2019}

One area not covered in this thesis but deserving of future research is the automation of social engineering through AI~\citep{mirsky_Threat_Offensive_AI_Organizations_2023}. However, current AI technology is not advanced enough to develop fully autonomous agents capable of executing attacks without human supervision and aid.



What seems certain is that we can count on the rapid development of AI technologies, AI-based social engineering attacks evolving with them, and the need for continuous, innovative user training growing in the future. Attackers and defenders are playing a never-ending game of "cat \& mouse" where nobody can rest.

It is likely that deepfake phishing incidents will become ever more prevalent \citep{mirsky_Threat_Offensive_AI_Organizations_2023} due to the technology being mature, harder to mitigate than regular phishing attacks, is more effective at trust exploitation, can expedite attacks and deepfakes is a new type of phishing tactic that not all cyber defenders are aware of.

Experts from both academia and industry should concentrate their efforts on deterring the top threats, namely social engineering powered by AI and impersonation with deepfakes \citep{mirsky_Threat_Offensive_AI_Organizations_2023}