
\chapter{Methods\label{methods}}

%% Käsittelylukujen työnjako määräytyy käsiteltävän asian luonteen mukaisesti. Lukijan oh￾jailemiseksi kukin pääluku kannattaa aloittaa lyhyellä kappaleella, joka paljastaa mikä kyseisen luvun keskeisin sisältö on ja kuinka aliluvuissa asiaa kehitellään eteenpäin. Er￾ityisesti kannattaa kiinnittää huomiota siihen, että lukijalle ilmaistaan selkeästi miksi kutakin asiaa käsitellään ja miten käsiteltävät asiat suhtautuvat toisiinsa. Jäsentelyongelmista kielivät tilanteet, joissa alilukuja on vain yksi, tai joissa käytetään useampaa kuin kahta tasoa (pääluku ja sen aliluvut). Kolmitasoisia otsikointeja saate￾taan tarvita joissakin teknisissä dokumenteissa perustellusti, mutta nämä muodostavat poikkeuksen.

%% Perusohjeena on käyttää tekstin rakenteellisesti painokkaita paikkoja, kuten lukujen avauk￾sia ja teksikappaleiden aloitusvirkkeitä juonenkuljetukseen ja informaatioaskeleiden sito￾miseen toisiinsa. Tekstikappaleiden keskiosat, samoin kuin lukujen keskiosat selostavat asiaa vähemmän tuntevalle yksityiskohtia, kun taas aihepiirissä jo sisällä olevat lukijat voivat alkuvirkkeitä silmäilemällä edetä tekstissä tehokkaasti eksymättä tarinan juonesta.

%% Kullakin kirjoittajalla on oma temponsa, joka välittyy lukijalle tekstikappaleiden pitu￾udessa ja niihin sisällytettyjen ajatuskulkujen mutkikkuudessa. Kussakin tekstikappaleessa pitäisi pitäytyä vain yhdessä informaatioaskelessa tai olennaisessa päättelyaskelessa, muuten juonen seuraaminen käy raskaaksi olennaisten lauseiden etsiskelyksi. Yksivirkkeisiä tek￾stikappaleita on syytä varoa.