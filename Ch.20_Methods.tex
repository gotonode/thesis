\chapter{Methods\label{methods}}

Bold text \textbf{here}



Itemized list here

\begin{enumerate}
    \item one\rightarrow a
    \item two
    \item three
\end{enumerate}

Bulleted
\begin{itemize}
    \item a
    \item b
    \item c
    \item d
    \item e
    \item f
    \item 

\begin{table}
    \centering
    \begin{tabular}{ccc}
         as&  hd& nc\\
         ,a&  & \\
         &  & \\
         &  & \\
    \end{tabular}
    \caption{My Table 2}
    \label{tab:my_label}
\end{table}
\end{itemize}

Social engineering exploits human psychology rather than technical hacking techniques to gain access to buildings, systems, or data. One common method is phishing, where attackers send fraudulent emails or messages that appear to be from a trusted source, aiming to trick individuals into revealing sensitive information. Pretexting is another technique where an attacker fabricates scenarios to engage a victim in a manner that increases the likelihood of sharing information or granting access to restricted areas. Tailgating, where an attacker follows authorized personnel into a restricted area without proper authentication, and baiting, where malware is hidden within a seemingly benign download or physical device like a USB drive, are also prevalent. These strategies leverage the natural human tendency to trust, be helpful, or react without suspicion, underscoring the importance of awareness and training in defending against social engineering threats \citep{neupane_social_2018}.
Test

More test

