

    %%%%%%%%%%%%%%%%%%%%%%%%%%%%%%%%%%%%%%%%%%%%%%%
    %% DEFINITION                                %%
    %%%%%%%%%%%%%%%%%%%%%%%%%%%%%%%%%%%%%%%%%%%%%%%
    

\chapter{History and Definition of Social Engineering\label{definition}}

The term \textit{social engineering} is perhaps overused and is certainly misused, and for clarity, in this chapter a clearer definition is formed and some history reviewed. For the purpose of this thesis, we'll use the definition for SE given by \cite{wang_defining_2020}: "social engineeing is a type of attack wherein the attacker(s) exploit human vulnerabilities by means of social interaction to breach cybersecurity, with or without the use of techincal means and technical vulnerabilities."

The term \textit{social engineering} seems to have first appeared in an article titled "More on Trashing", which was published in September, 1984 on one of the earliest hacker magazines, The Hacker's Quarterly \citep{wang_defining_2020}, but the broader concepts of human exploitation date back much farther than this \citep{qin_investigation_2007}. For as long as humans have partaken in communication and trade, there have been those that have tried to exploit the system in their own favor, in an unethical and selfish way.

% impersonation
% phishing
% baiting
% tailgating

For as long as there has been social contracts, both written and oral or implied, there have been those that try to get away with cheating. Social engineering, thus, is not a new concept. In 19xx people who partook in actions akin to SE were called confidence men or con artists. They manipulated their victim to be complacent in the act.

As described by \cite{abiteboul}, the term \textit{social engineering} is perhaps overused and is certainly misused. What exactly constitutes social engineering? In this thesis, SE is defined as "social engineering is the deliberate act of convincing a victim, usually though the use of technology, to perform an action that may or may not be in their best interest". A definition of social engineering by Hadnagy has become quite popular "may or may not be in their best interest". HADNAGY

Some scholars have included acts like shoulder shurfing and dumpster diving as social engineering attacks \citep{abiteboul}, even though they do not rely on the manipulation of individuals.  However, since they don't fall neatly into the typical category of "hacking attacks", they fall into a gray area in between. Since user education and awareness programs should include training the user against discarding important documents without shredding them first or by putting them into boxes designated "secure documents", and people need to be on the lookout for people gazing over their shoulders, especially when entering sensitive information such as usernames, passwords and other access codes, they are included in this thesis.

It's also necessary to go over some basic terminology related to the field of SE.

Two common attack methods, namely \textbf{dumpster diving} and \textbf{shoulder surfing}, are often categorized as social engineering attacks, but do not necessarily fall under the SE category, as they do not include direct social interaction with the victim \citep{wang_defining_2020}. In other words, the compliance of the victim is not necessary in these two types of attacks. However, since they are non-technical in that no devices or technologies need to be used, they fall in a "gray area". They require neither the use of devices or technologies nor the psychological manipulation of the user. Thus, it has been easy to refer to them as SE attacks. In this paper, they are still discussed because SE training and awareness programs should include mentioning about these.








        %%%%%%%%%%%%%%%%%%%%%%%%%%%%%%%%%%%%%%%%%%%%%%%
        %% Open-Source Intelligence (OSINT)          %%
        %%%%%%%%%%%%%%%%%%%%%%%%%%%%%%%%%%%%%%%%%%%%%%%

\section{Open-Source Intelligence, OSINT}

OSINT, sometimes written as OS-INT, means open-source intelligence. Like the name implies, it involves gathering of intelligence data from publicly locatable sources, such as from the target company's website, or from the social networking profiles of an individual or from other public records.

As people have adopted to using social media as sometimes their primary means of communication, a lot of exploitable data is shared on these platforms as well.