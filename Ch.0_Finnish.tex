
% --------------------------------
% FINNISH SUMMARY
% --------------------------------


\chapter*{Tekoälyä hyödyntävä\\käyttäjän manipulointi\label{chapter:finnish}}
\begin{comment}

Ohjeet:
    - 4 or 5 sivua

\end{comment}


Käyttäjän manipuloinnilla (\textit{social engineering}) tarkoitetaan tietoturvan kontekstissa tietojärjestelmän loppukäyttäjään kohdistuvaa tietoturvahyökkäystä.

\section*{Hyökkäykset}

Tässä luvussa käydään läpi hyökkäyksiä.

\section*{Puolustuskeinot}

Tässä luvussa käydään läpi puolustuskeinoja.

\section*{Puolustuskeinojen arviointia}

Tässä luvussa käydään arvointia puolustuskeinojen tehokkuudesta.

\section*{Yhteenveto}

Vaikuttaa siis siltä että voimme olettaa tekoälyjärjestelmien nopean kehittymisen jatkuvan, tietoturvauhkien kehittymisen niiden mukana, ja tarpeen jatkuvalle käyttäjien kouluttamiselle ja uusien puolustuskeinojen löytämiselle kasvavan.