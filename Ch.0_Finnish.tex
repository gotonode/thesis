
% --------------------------------
% FINNISH SUMMARY
% --------------------------------


\chapter*{Tekoälyavusteinen käyttäjän\\manipulointi\label{chapter:finnish}}
\begin{comment}

Tekoälyä hyödyntävä käyttäjän manipulointi
Teokälypohjainen käyttäjän manipulointi
Tekoälyavusteinen käyttäjän manipulointi

Pyydä Riinalta ym palautetta kieliopin tarkistuksessa! Opin samalla itse. Riinahan voi tarkistaa esim tätä .tex tiedostoa GitHubista? Tai PDF kumpi vaan hänelle parempi, mutta PDF:n kassa pitää muistaa aina päivittää se Overleafiin ja sitten GitHubiin.

Ohjeet:
    - 4 or 5 sivua
    - TOC ja Chapter 1 Introduction väliin

Kappaleet:
    - (ilman nimeä sisältää Introduction ja Definition kappaleet)
    - Hyökkäykset ja työkalut
    - Puolustuskeinot
    - Puolustuskeinojen arviointia
    - Yhteenveto
    - EI Overleaf kappalenumerointia? Kappale "0"?
    

\end{comment}


Käyttäjän manipuloinnilla (\textit{social engineering}) tarkoitetaan tietoturvan kontekstissa tietojärjestelmän loppukäyttäjään, eli ihmiseen, kohdistuvaa tietoturvahyökkäystä. Sen sijaan että hyökkääjät etsisivät teknisiä haavoittuvuuksia, he kohdistavat hyökkäykset ihmiseen käyttäen hyväksi psykologisia menetelmiä.

OpenAI julkaisi vuonna 2022 ChatGPT:n joka mullisti tavan jolla ihmiset käyttävät tekoälyäpalveluita. Se keräsi yli 100 miljoonaa käyttäjään ensimmäisen 2 kuukauden aikana. ChatGPT on ns. generatiivinen tekoäly joka on koulutettu suurella määrällä dataa ja joka pystyy luomaan transformaatio-prosessin (\textit{transformer}) uutta sisältöä, kuten tekstiä tai kuvia.

\section*{Hyökkäykset ja työkalut}

Tässä luvussa käydään läpi hyökkäyksiä.

Hyökkääjät ovat onnistuneet valjastamaan ChatGPT:n kaltaiset suuret kielimallit (\textit{large language model}) ohittamalla niiden kehittäjien niille asettamia rajoituksia.

Tunnetuin käyttäjän manipulointihyökkäys on \textbf{tietojenkalastelu} (\textit{phishing)}. Tietojenkalastelu on petollista toimintaa missä hyökkääjä esiintyy luotettavana tahona tavoitteenaan saada käyttäjältä luottamuksellisia tietoja kuten hänen salasanansa tai luottokorttinsa numeron.

\section*{Puolustuskeinot}

Tässä luvussa käydään läpi puolustuskeinoja.

\section*{Puolustuskeinojen arviointia}

Tässä luvussa käydään arvointia puolustuskeinojen tehokkuudesta.

\section*{Yhteenveto}

Vaikuttaa siis siltä että voimme olettaa tekoälyjärjestelmien nopean kehittymisen jatkuvan, tietoturvauhkien kehittymisen niiden mukana, ja tarpeen jatkuvalle käyttäjien kouluttamiselle ja uusien puolustuskeinojen löytämiselle kasvavan.