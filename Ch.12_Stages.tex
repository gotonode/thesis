

    %%%%%%%%%%%%%%%%%%%%%%%%%%%%%%%%%%%%%%%%%%%%%%%
    %% STAGES                                    %%
    %%%%%%%%%%%%%%%%%%%%%%%%%%%%%%%%%%%%%%%%%%%%%%%


\chapter{Anatomy of an Attack\label{stages}}
\begin{comment}

Guides:
    - About 2 pages

TODO:
    [ ] Cover how SE attacks are cyclical

What to cover:
    - The 4-Stage process of SE
        - Gather OSINT
        - Build relationship
        - Exploit relationship / launch the attack
        - Exit / cover traces
    - Cyclical nature of SE attacks
    
Literature:
    - 

\end{comment}

Social engineering attacks are usually deployed in multiple stages, with each stage building on the previous stages. These stages can also loop cyclically, where multiple rounds of various stages are used to attack an organization or an individual. Various scholars define these stages in different ways \citep{mouton_social_2016}, but they generally follow a 4-stage process. Mitcnick, for example, categories them as ABCD.

The 4-stage process used on this thesis is:

\begin{enumerate}
    \item Collect information about the target
    \item Develop a relationship with the target
    \item Exploit the gathered information and the built relationship by executing the attack
    \item Exit having cleaned any traces
\end{enumerate}

An attack against an individual or a corporation can happen in multiple iterations, with subsequent iterations building on from the information and resources (such as passwords, access badges) gathered from the previous.

Next, each of these four stages is examined in detail. Special emphasis is placed on how the emergence of modern AI technologies could impact, or has already impacted, these stages. A subsequent chapter will go over the different attack types, followed by a chapter that examines countermeasures against them.

In this section, each of the four stages of an attack are examined from the point of view of old school SE and as well as AI-powered SE. Further analysis about the interplay of these attacks is in the last chapter.




        %%%%%%%%%%%%%%%%%%%%%%%%%%%%%%%%%%%%%%%%%%%%%%%
        %% Intelligence Gathering                    %%
        %%%%%%%%%%%%%%%%%%%%%%%%%%%%%%%%%%%%%%%%%%%%%%%

\section{Intelligence Gathering}
\begin{comment}
    
    - OSINT has been defined before
    - Cover the phase of intel gathering more broadly than just OSINT
    - Following and observing people entering and exiting premises
    - Calling the company for more information (after OSINT)
    - Use of pretexting in intel gathering

\end{comment}


First stage is research or information collection. This relies on what's known as OSINT, or open-source intelligence. OSINT is any information that is publicly available via the Internet or other means and that doesn't require any breach of security to be accessed.

This intelligence is then used to formulate an attack plan against the target individual or corporation.



% gather info, understand vulnerabilities, habits, potential entry/exit points
% yrityksen nettisivut, some profiilit (myös vanhat), julkinen tieto esim yritystietokannat
% OSINT ulkopuolella myös phishing emails?
% fyysinen käyttäjän seuranta
% AI voi automatisoida tietojen keruuta, ja analysointia (myös suuuuret datamäärät), nopeuttaa ja tehostaa tiedonkeruuta ja itse hyökkäystä
% NLP voi auttaa parsimisessa



        %%%%%%%%%%%%%%%%%%%%%%%%%%%%%%%%%%%%%%%%%%%%%%%
        %% Developing the Relationship               %%
        %%%%%%%%%%%%%%%%%%%%%%%%%%%%%%%%%%%%%%%%%%%%%%%

\section{Developing the Relationship}
\begin{comment}
    
    - Engaging the target with the gathered info and fabricated pretext
    - Choosing the right timing
    - Practicing the engagement
    - Creating the strategy for engagement

\end{comment}


Once the attacker has gained information and formulated an attack plan based on it, he 

% luo strategia, suunnitelma miten käyttää hyödyksi havaittuja haavoittuvuuksia
% luo persona, pretext jne vakuuttava viesti
% oikean ajastuksen valinta
% oikeiden hyökkäysmetodien valinta

% AI voi auttaa hyvinkin vakuuttavien viestien kirjoittamisessa, myös persona kehittäminen
% deepfake teknologia, chatbotit
% AI voi analysoida ja ennustaa tehokkaimmat ajat ja metodit hyökkäykselle kerätyn datan perusteella
 




        %%%%%%%%%%%%%%%%%%%%%%%%%%%%%%%%%%%%%%%%%%%%%%%
        %% Exploiting the Victim                     %%
        %%%%%%%%%%%%%%%%%%%%%%%%%%%%%%%%%%%%%%%%%%%%%%%

\section{Exploiting the Victim}
\begin{comment}
    
    - 

\end{comment}











        %%%%%%%%%%%%%%%%%%%%%%%%%%%%%%%%%%%%%%%%%%%%%%%
        %% The Exit                                  %%
        %%%%%%%%%%%%%%%%%%%%%%%%%%%%%%%%%%%%%%%%%%%%%%%

\section{The Exit}
\begin{comment}
    
    - 

\end{comment}


Hadnagy et al. emphasize that in whatever you do, leave your target better off for having met with. This is in stark contrast to typical SE attacks.

Any bridges should not be burned but the relationship should be exploitable in the future as well

The SE is an iterative process, use what was found from previous attacks with the next one

In this chapter the 4-stage social engineering attack model was examined so that the SE attack methods can be reviewed in a proper context, which is done in the next chapter.