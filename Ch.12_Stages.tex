

    %%%%%%%%%%%%%%%%%%%%%%%%%%%%%%%%%%%%%%%%%%%%%%%
    %% STAGES                                    %%
    %%%%%%%%%%%%%%%%%%%%%%%%%%%%%%%%%%%%%%%%%%%%%%%


\chapter{Stages of Social Engineering Attacks\label{stages}}

Social engineering attacks are deployed in multiple stages. Various scholars define these stages a little differently, but they generally follow a pattern such as the one defined by \cite{salahdine_social_2019}:

\begin{enumerate}
    \item Collect information about the target
    \item Develop a relationship with the target
    \item Exploit the available information and the built relationship by executing the attack
    \item Exit without leaving or having cleaned any traces
\end{enumerate}

SE attacks can be detected, but they cannot be fully stopped \citep{wang_defining_2020}.

Next, each is these four stages is examined in detail.

\section{Intelligence Gathering \& OSINT}

First stage is research or information collection. This relies on what's known as OSINT, or open-source intelligence. OSINT is any information that is publicly available via the Internet or other means and that doesn't require any breach of security to be accessed.

\section{Developing the Relationship}

\section{Exploiting the Victim}

\section{The Exit}

Hadnagy et al. emphasize that in whatever you do, leave your target better off for having met with. This is in stark contrast to typical SE attacks.

Any bridges should not be burned but the relationship should be exploitable in the future as well

The SE is an iterative process, use what was found from previous attacks with the next one