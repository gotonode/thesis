

    %%%%%%%%%%%%%%%%%%%%%%%%%%%%%%%%%%%%%%%%%%%%%%%
    %% ATTACK METHODS                            %%
    %%%%%%%%%%%%%%%%%%%%%%%%%%%%%%%%%%%%%%%%%%%%%%%

\chapter{AI-Powered Attacks\label{attacks}}

Awareness of various social engineering (SE) attacks is crucial for professionals across all industries, not just those in cybersecurity. This chapter provides an overview of the most common SE attack methods.

%A later chapter will examine how modern AI augments these attacks, and how users of information systems need to be trained to counter these new, advanced threats.

Some attacks that are discussed here are not always considered a type of SE attack, specifically shoulder surfing and dumpster diving, as these do not require the co-operation of the victim \citep{wang_defining_2020}. However, since they are often used in conjunction with other SE attacks, and since training for against them is often included in SE training and awareness programs, they are explained here

%and because attackers are constantly improvising their attack methods, some manipulation used in conjunction with these methods, such as convincing employees that the company will destroy any documents put in the general waste in a safe way, may be used.

To better understand the threat posed by SE, it is essential to examine the diverse strategies employed by attackers. The following are some of the most common and effective SE attack methods. We'll also analyze how the emergence of modern AI technologies might, or already has, powered up these types of attacks.

Countermeasures against these attacks are examined in a later chapter.









        %%%%%%%%%%%%%%%%%%%%%%%%%%%%%%%%%%%%%%%%%%%%%%%
        %% Phishing                                  %%
        %%%%%%%%%%%%%%%%%%%%%%%%%%%%%%%%%%%%%%%%%%%%%%%

\section{Phishing}

As the quintessential SE attack, \textbf{phishing} is characterized by malicious attempts to gain sensitive information from unaware users, usually via email and by using spoofed websites that look like their authentic counterparts. Phishing has been around since 1996, when cybercriminals began using deceptive emails and websites to steal AOL (America Online) account information from gullible users \citep{wang_defining_2020}.

\textbf{Spear phishing} is a more targeted version of phishing, where attackers customize their deceptive emails to a target individual or organization. Unlike with generic phishing attempts, this type of phishing involves gathering detailed information about the victim, via OSINT or otherwise, such as their name, position and contacts to craft a convincing and personalized message \citep{salahdine_social_2019}. This tailored approach increases the likelihood of the victim falling for the scam.

Last on the list of phishing attacks is \textbf{whaling}. Whaling, also known as CEO fraud, is a highly targeted phishing attack aimed at high-profile individuals within an organization, such as executives or senior management, "the big whales" \citep{abraham_overview_2010}. The attackers carefully research their targets to create convincing and typically urgent messages that appear to come from trusted sources, often impersonating colleagues, business partners, or government agencies. The goal is often to authorize large financial transactions or to leverage the target's authority and access within the company.

Two additional types of phishing need to be addressed, and they are \textbf{vishing} or voice phishing and \textbf{smishing} or SMS phishing. Despite having complicated names, the idea behind them are quite simple.

Regular phishing doesn't usually require OSINT but spear phishing does.











        %%%%%%%%%%%%%%%%%%%%%%%%%%%%%%%%%%%%%%%%%%%%%%%
        %% Bating                                    %%
        %%%%%%%%%%%%%%%%%%%%%%%%%%%%%%%%%%%%%%%%%%%%%%%

\section{Baiting}

Baiting is a similar attack method to phishing, discussed above, but emphasizes luring the victim via enticement strategies \citep{conteh_cybersecurityrisks_2016, salahdine_social_2019}. This technique exploits the target's curiosity or greed to gain unauthorized access to resources or premises or to obtain sensitive information.

%%A common type of a baiting attack is what's known as a "USB drop" where the attacker loads malware or Trojan software into USB drives that are often tagged with an enticing title such as "Layoff Plans 2024" or "Sarah's private photos" and which are subsequently dropped at convenient locations for employees of the target company to find. These locations could be easily accessed, such as the company's vicinities, or more hard-to-reach places such as a protected parking lot or even employee restrooms or dining areas. A famous experiment in which 297 flash drives were dropped on university campuses concluded that the success rate (drives connected) was between 45 \% -- 98 \%, with the first drive connected within less than six minutes \cite{tischer_users_2016}.

%%A USB baiting experiment showed that 15 out of 20 USB drives were found by employees and all were plugged in to the company's computers \citep{wang_social_2021}.

%%OSINT can be useful here as well, with a bit of searching the tag that is attached to the thumbdrive could have a more interesting and suitable title. Attackers can also easily order USB thumbdrives, as an example, with custom logos printed on them from various online retailers for increased effect. The logo could even be from a competing company's one, if there's been heated rivalry between the two.

%% https://www.darkreading.com/perimeter/social-engineering-the-usb-way

%% online offers, free downloads, gifts, fake job posting
%% plays on human trust, safe looking items, greed, curiosity
%% fake software updates, malicious ads

%% seuraukset: malware, trojan, data breach, financial loss, identity theft
%% red flags? unsolicitied offers, too good to be true deals, unfamiliar sources











        %%%%%%%%%%%%%%%%%%%%%%%%%%%%%%%%%%%%%%%%%%%%%%%
        %% Pretexting                                %%
        %%%%%%%%%%%%%%%%%%%%%%%%%%%%%%%%%%%%%%%%%%%%%%%

\section{Pretexting}

Pretexting involves fabricating a story or a scenario, a \textbf{pretext}, that is plausible but fraudulent, to engage the target and extract information with \citep{conteh_cybersecurityrisks_2016}. This type of attack relies heavily on OSINT, or the gathered open-source intelligence, in assisting with the creation of the story.

Modern AI can assist with the deployment of the pretext by being a sparring partner to the attacker, giving a safe "sandbox" to try out potential attacks and find ways in which the target might react.

%%Situations can change quickly, and the attacker must be quick on their feet to respond to these changes. Training in multiple possible ways the target might respond or react is highly beneficial, such as threatening to call the police or the IT support.

%%A high level of confidence in the pretext and the pretexted role is often cited to be a necessity, however, someone may very well engage a pretext as a novice, nervous new employee, which suggests that rather than stating a high level of (outwardly visible) confidence as a neceissty, the better wording would be a suitable level of confidence. In all cases, the attacker must believe their own pretext and act out the attack like an actor in a movie, to "become the pretext".

%% obtain sensitive info, access to systems or locations
%% must be believable and relevant, should fit with the target's environment and expected interactions
%% kerro kuinka exploitatataan luottamusta
%% kommunikaatiotaidot, professionalism, itsevarmuus (sopiva taso, jos esim. pretextaa aloittelijaa itsevarmuus voikin olla alhaisempi)
%% tyypillisiä skenaarioita: co-worker, kuljetushtiö, security audit, IT tuki, law enforcement, 
%% mitä dataa haetaan: henk koht tiedot, pankkitiedot, kirjautumistiedot, joka päivä vaihtuva "PIN" koodi, insider info
%% suojautuminen: soittajan id vahvistus (varsinkin tuntemattomien soittajien), caller spoofing vahvistus, mitä tietoja saa antaa, kenelle ja millä edellytyksillä, 
%% pretexting on laitonta monessa paikkakunnassa, paikassa, SE white hat audit
%% oikean maailman incidentit












        %%%%%%%%%%%%%%%%%%%%%%%%%%%%%%%%%%%%%%%%%%%%%%%
        %% Tailgating                                %%
        %%%%%%%%%%%%%%%%%%%%%%%%%%%%%%%%%%%%%%%%%%%%%%%

\section{Tailgating}

%% STATUS: kesken

\textbf{Tailgating}, also known as \textbf{piggybacking}, is a social engineering tactic that involves following an authorized person through an access-controlled passage, such as a security gate. This type of attack exploits individuals with temporary access rights, such as delivery personnel or maintenance workers \citep{conteh_cybersecurityrisks_2016}. The attacker may use manipulation to gain access, for instance, by carrying a heavy object and asking for assistance or pretending to be a delivery person with a forged pretext.

%%The attacker may have found out through OSINT that the company is expecting a delivery and then don the suit of a delivery company and escort people going in from their smoking/coffee breaks outside. Another type is where the attacker has forged a pretext where he explains that he had "forgotten" his security ID inside and his boss is already very angry with him and that he's afraid he's going to lose his job if he keeps messing up like this, playing on people's sympathy.

In another scenario, the attasacker may use psychological manipulation to evoke sympathy, claiming to have forgotten their security ID and worrying about losing their job. This tactic preys on people's natural instinct to help and be polite. The mundane routine of passing through security gates can also lead to a false sense of security, making individuals less vigilant.

%%This type of attack exploits people's natural tendency to want to help and be polite. Going in through security gates is a daily chore for many, which due to having become a mundane activity, has lessened the individual's vigilance.

%% may use machine learning algorithms to identify vulnerabilites in access control systems, making tailgating attacks even more dangerous












        %%%%%%%%%%%%%%%%%%%%%%%%%%%%%%%%%%%%%%%%%%%%%%%
        %% Shoulder Surfing                          %%
        %%%%%%%%%%%%%%%%%%%%%%%%%%%%%%%%%%%%%%%%%%%%%%%

\section{Shoulder Surfing}

%% STATUS: almost complete, need sources

Observing people enter sensitive information, such as login details or financial data, without their knowledge or approval is called \textbf{shoulder surfing}. The name implies a person watching over the shoulder of someone when they are typing or viewing sensitive data, but the attack surface is actually far larger. The attacker could use cameras, either hacked, or those placed there by the attacker for this purpose, or even everyday objects such as binoculars.

The proliferation of high-resolution cameras, such as those with HD or 4K resolution, has exacerbated the issue of unauthorized information gathering since security cameras of the past didn't have the resolution necessary to show what's being viewed on a screen. Going through tens or even hundreds of hours of video material where sensitive information might be visible on an individual's or an employee's screen is time-consuming and tedious, but with the help of AI this job can be carried out with more ease. AI can not only turn words in a video into text, but it can also summarize the found text and search for anything that could be exploited.

%% vigilance in public places
%% HD and 4K cameras
%% social norms, vigilance
%% pankkiautomaatti
%% kohteet PIN, salasanat, turvakoodit, henk koht tiedot
%% puhelimet, draw a symbol to access, halvat vakoilulaitteet aliexpress
%% privacy screens, HP SureView
%% toimiston layout joka hankaloittaa shoulder surfing
%% etätyö













        %%%%%%%%%%%%%%%%%%%%%%%%%%%%%%%%%%%%%%%%%%%%%%%
        %% Dumpster Diving                           %%
        %%%%%%%%%%%%%%%%%%%%%%%%%%%%%%%%%%%%%%%%%%%%%%%

\section{Dumpster Diving}

%% STATUS: lähes valmis

As the name implies, \textbf{dumpster diving} refers to the practice of going through an organization's or an individual's trash in order find sensitive information that should have been disposed of properly \citep{syafitri_social_2022}. Rummaged content may yield interesting results, typically in the form of documents such as papers but also media devices such as optical discs, hard drives or USB thumbdrives, to be used as-is or as part of a future SE attack.

Since dumpster diving does not include manipulation of people in its purest form, rather relying on the improper care of documents and other material, it is sometimes not classified as a social engineering attack \citep{wang_defining_2020}. However, it is often considered a precursor or supplementary tactic in broader social engineering schemes, as the information gathered can be used to craft more convincing and targeted attacks.

%% However, one could imagine scenarios where manipulation of people is used as part of a dumpster diving attack, for example convincing people that documents put into regular waste bins will be shredded as part of a new company policy, or that all discs and USB drives are constantly being encrypted and thus they can be safely discarded in the regular trash.









%% Käsittelylukujen työnjako määräytyy käsiteltävän asian luonteen mukaisesti. Lukijan oh￾jailemiseksi kukin pääluku kannattaa aloittaa lyhyellä kappaleella, joka paljastaa mikä kyseisen luvun keskeisin sisältö on ja kuinka aliluvuissa asiaa kehitellään eteenpäin. Er￾ityisesti kannattaa kiinnittää huomiota siihen, että lukijalle ilmaistaan selkeästi miksi kutakin asiaa käsitellään ja miten käsiteltävät asiat suhtautuvat toisiinsa. Jäsentelyongelmista kielivät tilanteet, joissa alilukuja on vain yksi, tai joissa käytetään useampaa kuin kahta tasoa (pääluku ja sen aliluvut). Kolmitasoisia otsikointeja saate￾taan tarvita joissakin teknisissä dokumenteissa perustellusti, mutta nämä muodostavat poikkeuksen.

%% Perusohjeena on käyttää tekstin rakenteellisesti painokkaita paikkoja, kuten lukujen avauk￾sia ja teksikappaleiden aloitusvirkkeitä juonenkuljetukseen ja informaatioaskeleiden sito￾miseen toisiinsa. Tekstikappaleiden keskiosat, samoin kuin lukujen keskiosat selostavat asiaa vähemmän tuntevalle yksityiskohtia, kun taas aihepiirissä jo sisällä olevat lukijat voivat alkuvirkkeitä silmäilemällä edetä tekstissä tehokkaasti eksymättä tarinan juonesta.

%% Kullakin kirjoittajalla on oma temponsa, joka välittyy lukijalle tekstikappaleiden pitu￾udessa ja niihin sisällytettyjen ajatuskulkujen mutkikkuudessa. Kussakin tekstikappaleessa pitäisi pitäytyä vain yhdessä informaatioaskelessa tai olennaisessa päättelyaskelessa, muuten juonen seuraaminen käy raskaaksi olennaisten lauseiden etsiskelyksi. Yksivirkkeisiä tek￾stikappaleita on syytä varoa.