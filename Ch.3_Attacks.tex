


% --------------------------------
% ATTACK VECTORS AND TOOLS
% --------------------------------

\chapter{Attack Vectors and Tools \label{chapter:attacks}}
\begin{comment}

Guides:
    - About 3-4 pages

TODO:
    [ ] 

What to cover:
    - Attacks
        - Deepfake generated synthetic media
            - Videos
            - Images
            - Audio
            - Real-time voice morphing

Sections:
    - Attack Vectors and Tools
        - Chatbots
        - Automated intelligence gathering
        - Deepfake-generated media
        - Phishing & spear phishing

\end{comment}

This chapter reviews key social engineering attack vectors and tools relevant to the modern threat of generative AI. It first explores the misuse of chatbots like ChatGPT for content generation, followed by an investigation of automated intelligence gathering processes. The discussion then covers deepfake-generated media that can be used for impersonation, concluding with how to use all of this with spear phishing.

This chapter reviews key social engineering attack vectors and tools relevant to the modern threat of generative AI. It first explores the misuse of chatbots like ChatGPT for content generation, followed by automated intelligence gathering processes. The discussion then covers phishing and spear phishing, concluding with deepfake-generated media.

This chapter provides an overview of some of the most common social engineering attack methods, paying attention to how modern AI technologies are or could be augmenting them. After that, Chapter \ref{chapter:countermeasures} goes over the countermeasures against these attacks.













% --------------------------------
% Chatbots
% --------------------------------

\section{Chatbots}

\begin{comment}

What to cover:
    - Mitä ovat chatbotit kuten ChatGPT
    - How Generative AI can be used by both cybersecurity professionals and threat actors
    - Circumventing ChatGPT's ethical restrictions with, for example prompt injections attacks or reverse psychology (with at least 1-2 examples)
    - How scholars and regular users have found ways to bypass ChatGPT's ethical restrictions??
    - Tekoälyn päivitys kun löydetään uusia tapoja ohittaa sen eettiset ohjeistukset ja kehittäjien asettamat rajoitukset
    - Pyydetään tekoälyä roolipelaamaan social engineering skenaarioita
    - Kielioppi ja kirjoitusvirheiden korjaus scam viesteissä
    - Generation of malware?
    - Analysis of gathered data to find insights to be used against the victim
    
\end{comment}

Generative AI's (GenAI) can be used by malicious actors in their schemes, but due to the manufacturer's set limits, some workarounds need to be used \citep{guptaFromChatGPTtoThreatGPT2023}. Asking ChatGPT to provide links to websites which provide pirated content such as movies results in ChatGPT denying the request, stating that downloading pirated content is unethical and may also lead the user's computer to be infected with malware.

% ChatGPT reached 100 million users in 2 months https://explodingtopics.com/blog/chatgpt-users

Regular users and scholars have found a number of ways to bypass ChatGPT's inherent ethic and behavioral guidelines, such as by using reverse psychology \citep{guptaFromChatGPTtoThreatGPT2023}. Instead of directly asking for links to the pirate websites, the user can say that because he doesn't want his computer to be infected by malware, ChatGPT should provide links to sites the user should avoid visiting, thus causing ChatGPT to reveal the content the user originally wanted.














% --------------------------------
% Automated intelligence gathering
% --------------------------------

\section{Automated intelligence gathering}

\begin{comment}

Automated intelligence gathering

What to cover:
    - What is automated intelligence gathering
    - Työkalut kuten Maltego (etsi muitakin työkaluja jos teet listan näistä, varmista netistä et ne on top 3 työkalut)
    - iCloner siitä yhdestä artikkelista All Your Contacts Are Blong To Us?

\end{comment}

Automated intelligence gathering has emerged as a pivotal elemental in contemporary information dissemination and strategic decision-making. By utilizing sophisticated algorithms and machine learning techniques, threat actors can efficiently collect, process and analyze vast amounts of data from various source, including social media, news outlets and public databases \citep{bilgeAllYourContactsAreBelongToUs2009}. This process only not streamlines the aquisition of relevant data but also enhances the accuracy and timeliness of insights derived from the data.

This gathered data can then be used with LLMs such as ChatGPT to generated highly convincing and personalized spear phishing messages, which are discussed next.

An advanced level of personalization is reached automatically through data mining and analysis where AI processes through vast amounts of publicly available information on social media platforms such as Facebook, X (Twitter) and Instagram, on forums and other digital resources to extract insights about potential victims. 














% --------------------------------
% Deepfake-generated content
% --------------------------------

\section{Deepfake-generated media}
\begin{comment}
Deepfake-generated content

What to cover:
    - What is a deepfake
    - Deepfakeja ei käsitely aiemmin? Generative AI kappaleessa?
    - Seuraava section kertoo tietojenkalastelusta ja sitoo chatbotit, automated intelligence gathering ja nää deepfaket yhteen kokonaisuudeksi

\end{comment}

\textbf{Deepfake}, a portmanteau of "deep learning", a type of machine learning, and "fake", is technology which uses AI to create highly convincing fake media, either by altering existing content or creating them from scratch \citep{mirskyTheCreationAndDetectionOfDeepfakes2021}. Deepfake content can be images, audio, and even full-resolution video.

By utilizing deepfake-generated content, deepfakes, attackers can convincingly impersonate trusted individuals or organizations, enhancing the credibility and even the emotional impact of their deceptive strategies. For example, a deepfake video of the victim's company's CEO making an urgent request for sensitive information can exploit the employee's natural tendencies to comply with authority, thus bypassing any skepticism that could've risen from a simple email message.

These deepfakes are then delivered to the victim via a number of different channels, such as email, instant messaging, SMS messages or phone/VoIP (Voice over IP) along with any other relevant information pertaining to the attacker's attempt at influence and manipulation.














% --------------------------------
% Phishing & spear phishing
% --------------------------------

\section{Phishing \& spear phishing}
\begin{comment}
Phishing & spear phishing

What to cover:
    - What is phishing (via email and ALSO other means)
    - Spear phishing a more targeted form of phishing
    - How ChatGPT can be used to improve scam messages
    - ChatGPT:n eettisten ohjeistusten ohittaminen on jo käsitelty kohdassa Chatbots

\end{comment}

As the quintessential social engineering attack, \textbf{phishing} is characterized by malicious attempts to gain sensitive information from unaware users, usually via email and by using spoofed websites that look like their authentic counterparts \citep{basitComprehensiveSurveyAIenabledPhishingAttacks2021}. Phishing has been around since 1996, when cybercriminals began using deceptive emails and websites to steal AOL (America Online) account information from unsuspecting users \citep{wangDefiningSocialEngineering2020}.

\textbf{Spear phishing} is a more targeted version of phishing, where attackers customize their deceptive messages to a target individual or organization \citep{basitComprehensiveSurveyAIenabledPhishingAttacks2021}. Unlike with generic phishing attempts, this type of phishing involves gathering detailed information about the victim, via OSINT or otherwise, such as their name, position and contacts to craft a convincing and personalized message \citep{salahdineSocialEngineeringAttacks2019}. This tailored approach increases the likelihood of the victim falling for the phishing attempt, but has traditionally been a lot more time and energy consuming.

By employing AI-powered techniques, attackers can automate the creation of deceptive spam messages, greatly enhancing the scale and precision of phishing attacks. AThese insights, such as by expressing in an email message the hope that the victim enjoyed the private company picnic last month and the caffeine-free sodas that were on offer, is used in the generation of the spear phishing messages to increase their seeming authenticity.

A hallmark of phishing messages has been the numerous and sometimes very obvious spelling and grammatical errors \citep{basitComprehensiveSurveyAIenabledPhishingAttacks2021}. ChatGPT can translate text from the attacker's native language to the victim's language without any loss of fidelity.