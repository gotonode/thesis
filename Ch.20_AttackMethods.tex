
\chapter{Attack Methods\label{methods}}

It's crucial for any professional working in any industry to be aware of the different SE attacks, not just for the cybersecurity professional. In this chapter, we'll go through the most common ones.

In a later chapter, we'll go through and analyze how modern AI augments these attacks, and how users of information systems need to be trained to counter these new, advanced threats.

SE attacks can be roughly taxonomized in the following way.


Some attacks that are discussed here are not always considered a type of SE attack, specifically shoulder surfing and dumpster diving, as these do not require the co-operation of the victim \citep{wang_defining_2020}. However, since they are often used in conjunction with other SE attacks, and since training for against them is often included in SE training and awareness programs, they are explained here.

\section{Phishing, Spear Phishing and Whaling}

\textbf{Phishing} is the quintessial social engineering attack. It is characterized by malicious attemps to gain sensitive information from unsuspecting users, usually via email and by using spoofed websites that look like their authentic counterparts. Phishing has been around since 1996, when cybercriminals began using deceptive emails and websites to steal AOL (America Online) account information from unsuspecting users. However, the concept of tricking people goes a lot further than that, with people doing such deeds being called confidence men or "con artists". 

\textbf{Spear phishing} is a more targeted version of phishing, where attackers customize their deceptive emails to a target individual or organization. Unlike with generic phishing attemps, this type of phishing involves gathering detailed information about the victim, such as their name, position and contacts to craft a convincing and personalized message. This tailored approach increases the likelyhood of the victim falling for the scam.

Last on our list of phishing attacks is \textbf{whaling}. Whaling, also known as CEO fraud, is a highly targeted phishing attack aimed at high-profile individuals within an organization, such as executives or senior management. The attackers careflly research their targets to create convinving and often urgent messages that appear to come from trusted sources, often impersonating colleagues, business partners, or government agencies. The goal is often to authorize large financial transactions or to leverage the target's authority and access within the company.

\section{Baiting}

Baiting is a similar attack method to phishing, discussed above, but emphasizes luring the victim via enticement strategies \citep{conteh_cybersecurityrisks_2016, salahdine_social_2019}. This technique exploits the target's curiosity or greed to gain unauthorized access to resources or premises or to obtain sensitive information.

A common type of a baiting attack is what's known as a "USB drop" where the attacker loads malware or Trojan software into USB drives that are often tagged with an enticing title such as "Layoff Plans 2024" or "Sarah's private photos" and which are subsequently dropped at convenient locations for employees of the target company to find. A USB baiting experiment showed that 15 out of 20 USB drives were found by employees and all were plugged in to the company's computers \citep{wang_social_2021}.

%% https://www.darkreading.com/perimeter/social-engineering-the-usb-way

%% online offers, free downloads, gifts, fake job posting



\section{Pretexting}

Pretexting involves fabricating a story or a scenario, a \textit{pretext}, that is plausible but fraudulent, to engage the target and extract information with \citep{conteh_cybersecurityrisks_2016}. This type of attack relies heavily on OSINT, or the gathered open-source intelligence, in assisting with the creation of the story.

Situations can change quickly, and the attacker must be quick on their feet to respond to these changes. Training in multiple possible ways the target might respond or react is highly beneficial, such as threatening to call the police or the IT support.

A high level of confidence in the pretext and the pretexted role is often cited to be a necessity, however, someone may very well engage a pretext as a novice, nervous new employee, which suggests that rather than stating a high level of (outwardly visible) confidence as a neceissty, the better wording would be a suitable level of confidence. In all cases, the attacker must believe their own pretext and act out the attack like an actor in a movie, to "become the pretext".

%% obtain sensitive info, access to systems or locations
%% must be believable and relevant, should fit with the target's environment and expected interactions
%% kerro kuinka exploitatataan luottamusta
%% kommunikaatiotaidot, professionalism, itsevarmuus (sopiva taso, jos esim. pretextaa aloittelijaa itsevarmuus voikin olla alhaisempi)
%% tyypillisiä skenaarioita: co-worker, kuljetushtiö, security audit, IT tuki, law enforcement, 
%% mitä dataa haetaan: henk koht tiedot, pankkitiedot, kirjautumistiedot, joka päivä vaihtuva "PIN" koodi, insider info
%% suojautuminen: soittajan id vahvistus (varsinkin tuntemattomien soittajien), caller spoofing vahvistus, mitä tietoja saa antaa, kenelle ja millä edellytyksillä, 
%% pretexting on laitonta monessa paikkakunnassa, paikassa, SE white hat audit
%% oikean maailman incidentit

\section{Tailgating}

Tailgating refers to the act of "tailing" someone through a access-controlled passage, such as a security gate. The social engineer may hold something that looks heavy in their hands and ask for a person going in to let them through, thus bypassing the need to have valid access authority \citep{conteh_cybersecurityrisks_2016}. This type of attack exploits those who already have access rights for temporary access, such as for the fabricated scenario of delivering new hardware to the server rooms. The attacker may have found out through OSINT that the company is expecting a delivery, and then don the suit of a delivery company and escort people going in from their smoking/coffee breaks outside.




\section{Shoulder Surfing}

Two common attack methods, namely \textbf{dumpster diving} and \textbf{shoulder surfing}, are often categorized as social engineering attacks, but do not necessarily fall under the SE category, as they do not include direct social interaction with the victim \citep{wang_defining_2020}. In other words, the compliance of the victim is not necessary in these two types of attacks. However, since they are non-technical in that no devices or technologies need to be used, they fall in a "gray area". They require neither the use of devices or technologies nor the psychological manipulation of the user. Thus, it has been easy to refer to them as SE attacks. In this paper, they are still discussed because SE training and awareness programs should include mentioning about these.

\section{Dumpster Diving}

As the name implies, \textbf{dumpster diving} refers to going through an organization's or an individual's trash, usually paper trash only but sometimes all trash, in order find sensitive information that really should've been disposed of properly. Since dumpster diving doesn't include manipulation of people in its purest form, rather relying on the improper care of documents and other material, it's sometimes not classified as a social engineering attack.


%% Käsittelylukujen työnjako määräytyy käsiteltävän asian luonteen mukaisesti. Lukijan oh￾jailemiseksi kukin pääluku kannattaa aloittaa lyhyellä kappaleella, joka paljastaa mikä kyseisen luvun keskeisin sisältö on ja kuinka aliluvuissa asiaa kehitellään eteenpäin. Er￾ityisesti kannattaa kiinnittää huomiota siihen, että lukijalle ilmaistaan selkeästi miksi kutakin asiaa käsitellään ja miten käsiteltävät asiat suhtautuvat toisiinsa. Jäsentelyongelmista kielivät tilanteet, joissa alilukuja on vain yksi, tai joissa käytetään useampaa kuin kahta tasoa (pääluku ja sen aliluvut). Kolmitasoisia otsikointeja saate￾taan tarvita joissakin teknisissä dokumenteissa perustellusti, mutta nämä muodostavat poikkeuksen.

%% Perusohjeena on käyttää tekstin rakenteellisesti painokkaita paikkoja, kuten lukujen avauk￾sia ja teksikappaleiden aloitusvirkkeitä juonenkuljetukseen ja informaatioaskeleiden sito￾miseen toisiinsa. Tekstikappaleiden keskiosat, samoin kuin lukujen keskiosat selostavat asiaa vähemmän tuntevalle yksityiskohtia, kun taas aihepiirissä jo sisällä olevat lukijat voivat alkuvirkkeitä silmäilemällä edetä tekstissä tehokkaasti eksymättä tarinan juonesta.

%% Kullakin kirjoittajalla on oma temponsa, joka välittyy lukijalle tekstikappaleiden pitu￾udessa ja niihin sisällytettyjen ajatuskulkujen mutkikkuudessa. Kussakin tekstikappaleessa pitäisi pitäytyä vain yhdessä informaatioaskelessa tai olennaisessa päättelyaskelessa, muuten juonen seuraaminen käy raskaaksi olennaisten lauseiden etsiskelyksi. Yksivirkkeisiä tek￾stikappaleita on syytä varoa.