
\chapter{Methods\label{methods}}

It's crucial for any professional working in any industry to be aware of the different social engineering (SE) attacks. In this chapter, we'll go through the most common ones.

In a later chapter, we'll go through and analyze how modern AI augments these attacks, and how users of information systems need to be trained to counter these new, advanced threats.

As described by \cite{abiteboul}, the term \textit{social engineering} is perhaps overused and is certainly misused. What exactly constitutes social engineering? In this paper, we'll use the defition of X by Y, "social engineering is the deliberate act of convincing a victim, usually though the use of technology, to perform an action that may or may not be in their best interest". Some have included acts like shoulder shurfing and dumpster diving as social engineering attacks, but since these do not rely on the manipulation of people we'll leave them outside the scope of this paper.

\section{Phishing, Spear Phishing and Whailing}

\textbf{Phishing} is the quintessial social engineering attack. It is characterized by malicious attemps to gain sensitive information from unsuspecting users, usually via email and by using spoofed websites that look like their authentic counterparts. Phishing has been around since 1996, when cybercriminals began using deceptive emails and websites to steal AOL (America Online) account information from unsuspecting users. However, the concept of tricking people goes a lot further than that, with people doing such deeds being called confidence men or "con artists". 

\textbf{Spear phishing} is a more targeted version of phishing, where attackers customize their deceptive emails to a target individual or organization. Unlike with generic phishing attemps, this type of phishing involves gathering detailed information about the victim, such as their name, position and contacts to craft a convincing and personalized message. This tailored approach increases the likelyhood of the victim falling for the scam.

Last on our list of phishing attacks is \textbf{whaling}. Whaling, also known as CEO fraud, is a highly targeted phishing attack aimed at high-profile individuals within an organization, such as executives or senior management. The attackers careflly research their targets to create convinving and often urgent messages that appear to come from trusted sources, often impersonating colleagues, business partners, or government agencies. The goal is often to authorize large financial transactions or to leverage the target's authority and access within the company.

\section{Pretexting}

Pretexting, a term first used by FBI in YYY to descibe a situation where a fabricated story (a pretext) is used to  lure

\section{Tailgating}

Tailgating refers to the act of "tailing" someone through a access-controlled passage, such as a security gate. The social engineer may hold something that looks heavy in their hands and ask for a person going in to let them through, thus bypassing the need to have a valid access authority



%% Käsittelylukujen työnjako määräytyy käsiteltävän asian luonteen mukaisesti. Lukijan oh￾jailemiseksi kukin pääluku kannattaa aloittaa lyhyellä kappaleella, joka paljastaa mikä kyseisen luvun keskeisin sisältö on ja kuinka aliluvuissa asiaa kehitellään eteenpäin. Er￾ityisesti kannattaa kiinnittää huomiota siihen, että lukijalle ilmaistaan selkeästi miksi kutakin asiaa käsitellään ja miten käsiteltävät asiat suhtautuvat toisiinsa. Jäsentelyongelmista kielivät tilanteet, joissa alilukuja on vain yksi, tai joissa käytetään useampaa kuin kahta tasoa (pääluku ja sen aliluvut). Kolmitasoisia otsikointeja saate￾taan tarvita joissakin teknisissä dokumenteissa perustellusti, mutta nämä muodostavat poikkeuksen.

%% Perusohjeena on käyttää tekstin rakenteellisesti painokkaita paikkoja, kuten lukujen avauk￾sia ja teksikappaleiden aloitusvirkkeitä juonenkuljetukseen ja informaatioaskeleiden sito￾miseen toisiinsa. Tekstikappaleiden keskiosat, samoin kuin lukujen keskiosat selostavat asiaa vähemmän tuntevalle yksityiskohtia, kun taas aihepiirissä jo sisällä olevat lukijat voivat alkuvirkkeitä silmäilemällä edetä tekstissä tehokkaasti eksymättä tarinan juonesta.

%% Kullakin kirjoittajalla on oma temponsa, joka välittyy lukijalle tekstikappaleiden pitu￾udessa ja niihin sisällytettyjen ajatuskulkujen mutkikkuudessa. Kussakin tekstikappaleessa pitäisi pitäytyä vain yhdessä informaatioaskelessa tai olennaisessa päättelyaskelessa, muuten juonen seuraaminen käy raskaaksi olennaisten lauseiden etsiskelyksi. Yksivirkkeisiä tek￾stikappaleita on syytä varoa.