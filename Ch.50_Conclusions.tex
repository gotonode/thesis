

    %%%%%%%%%%%%%%%%%%%%%%%%%%%%%%%%%%%%%%%%%%%%%%%
    %% CONCLUSIONS                               %%
    %%%%%%%%%%%%%%%%%%%%%%%%%%%%%%%%%%%%%%%%%%%%%%%


\chapter{Conclusions\label{chapter:conclusions}}
\begin{comment}
    
Guides:
    - As many pages as it takes (thesis max is around 20 pages)

TODO:
    [ ] How AI has augmented SE attacks and countermeasures
    [ ] Gap in the literature

What to cover:
    - How AI has augmented SE attacks and countermeasures
    - Gap in the literature regarding SE and AI intersection?
    - Analysis on where AI-powered SE attacks might be headed in the future
        - Also about robotics and human-like actors
    - What organizations and individuals need to do regarding the evolving landscape of SE attacks

Speculation:
    - Drones dropping USB thumbdrives?
    - Human-like android as threat actors
    - Impact of robotics on dumpster diving, shoulder surfing and baiting
    
Literature:
    - Gen and detection of deepfakes

From training material:
    - Yhteenveto vaatimattomimmillaan on vain lyhyt kertaus kirjoituksen keskeisistä asioista. Arvokkaamman yhteenvedon saa aikaan kommentoimalla työn tulosten arvoa, työn liittymistä ympäristöön ja tulevaisuudennäkymiä. Tällaiset arviot huolellisesti perusteltava.

\end{comment}

The landscape of social engineering (SE) has undergone a significant transformation with the advent of modern artificial intelligence. This thesis explored the dual-faceted impact of AI on SE, highlighting both the enhanced capabilities it provides to malicious actors and the advanced countermeasures it provides for cybersecurity professionals.

AI has empowered attackers to execute highly sophisthicated and targeted social engineering attacks, such as phishing, vishing (voice phishing), and deepfake-generated content, with truly alargming precision. These AI-powered attacks exploit human vulnerabilities more efficiently than ever before, making traditional detection and prevention methods increasingly inadequate.

AI's role in augmenting social engineering attacks is most evident in its ability to automate and amplify deceptive practices. Techniques such as Natural Language Processing (NLP, not to be confused with Neuro-Linguisting Processing) allow for the generation of personalized phishing messages that are difficult to distinguish from legitimate communications. Deepfake technologies create convincing syntehtic media, including images, videos and real-time voice morphing, which can be used to deceive targets into divulging sensitive information or performing actions on the attacker's request.

Automated open-source intelligence (OSINT) gathering further enchances the attacker's ability to craft belieavable pretexts and scenarios, making their social engineering effors more convincing and harder to detect.

For an example, where previously an employee could authenticate a caller by recognizing their voice \citep{mitnickArtDeceptionControlling2003}, intonations, and accents, today and especially in the near future this will not be enough. User training and awareness programs need to be updated for novel threat of AI in SE.

%As we've seen, SE is still as much a threat as it has ever been, despite major efforts to the contrary.

What's certain is that we can count on the rapid development of AI technologies, AI-based social engineering attacks evolving with it, and the need for continuous, innovative user training growing in the future. Attackers and defenders are playing a never-ending game of "cat \& mouse" where nobody can rest.

%X in Y references that training users effects will wear off in 3 weeks, necessiting continous retraining approaches.

%I'll end with the question that I started with; what, if anything, can the end-user trust anymore? And perhaps, with the advances in AI technology, the answer is "no-one".


%Modern AI can assist with the deployment of the pretext by being a sparring partner to the attacker, giving a safe "sandbox" to try out potential attacks and find ways in which the target might react.

% calling a CEO to get their voice sample, if not found via OSINT from YouTube or TV news or elsewhere etc