

    %%%%%%%%%%%%%%%%%%%%%%%%%%%%%%%%%%%%%%%%%%%%%%%
    %% CONCLUSIONS                               %%
    %%%%%%%%%%%%%%%%%%%%%%%%%%%%%%%%%%%%%%%%%%%%%%%


\chapter{Conclusions\label{conclusions}}

As we've seen, SE is still as much a threat as it has ever been, despite major efforts to the contrary.

What's certain is that we can count on AI developing, AI-based social engineering attacks evolving with it, and the need for continuous, innovative user training growing in the future. Attackers and defenders are playing a never-ending game of "cat \& mouse" where nobody can rest.

X in Y references that training users effects will wear off in 3 weeks, necessiting continous retraining approaches.

I'll end with the question that I started with; what, if anything, can the end-user trust anymore? And perhaps, with the advances in AI technology, the answer is "no-one".

% advancements in robotics could make baiting, tailgating, dumpster diving and shoulder surfing more lucrative in the future
% drones dropping USB thumbdrives (like they do in North Korea)

% Yhteenveto vaatimattomimmillaan on vain lyhyt kertaus kirjoituksen keskeisistä asioista. Arvokkaamman yhteenvedon saa aikaan kommentoimalla työn tulosten arvoa, työn liit￾tymistä ympäristöön ja tulevaisuudennäkymiä. Tällaiset arviot huolellisesti perusteltava.